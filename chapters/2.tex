


\chapter{相关理论基础}
\thispagestyle{others}
\pagestyle{others}
\xiaosi

\section{本章引言}

遥感图像云去除任务涉及复杂的成像退化过程与信息缺失问题,其重建效果不仅受制于云与大气对光学成像的物理干扰机理,还与不同遥感数据源的信息表达特性及模型建模能力密切相关。为从理论层面支撑后续方法设计,有必要对云遮挡条件下的遥感图像退化特性、多源遥感数据的成像机理差异,以及深度学习模型在图像重建任务中的建模特性进行系统梳理。同时,明确遥感图像重建质量的评价依据,也是开展定量实验分析的重要前提。基于此,本章围绕遥感图像云去除所涉及的相关理论基础展开,为后文提出的方法模型的提出奠定理论基础。

\section{光学遥感图像退化原理}

在光学遥感成像过程中,传感器所获取的影像并非地表真实反射特性的直接记录,而是地表辐射信号在穿过云层与大气介质后形成的观测结果。当成像路径中存在云或复杂大气条件时,地表辐射信号在传播过程中会发生显著改变,导致观测影像与真实地表之间产生偏差。这种偏差在遥感影像中表现为信息退化,其结果不仅体现在图像质量下降,更重要的是造成地表信息在空间结构和光谱维度上的缺失。本节将从云遮挡和大气传输两个方面,对遥感图像退化产生的原因、作用机制及其带来的信息缺失后果进行系统分析。

\subsection{云遮挡对光学遥感成像影响}

云遮挡是导致光学遥感图像退化的主要因素之一,其本质原因在于云层对电磁辐射传播路径的直接阻断。光学遥感成像依赖地表目标对太阳辐射的反射信号,而当云层位于传感器与地表之间时,地表反射辐射在传播过程中会被云体大量反射和散射,难以有效到达传感器。此时,传感器接收到的辐射信号主要来源于云体本身,而非真实地表目标。

(TODO:传感器、云层、地表反射的图,以及云污染图像)

从成像机制角度看,云遮挡引发的退化并非简单的亮度变化,而是导致地表信息在观测层面被直接“替换”。在厚云覆盖区域,来自地表的辐射信号几乎完全被云层屏蔽,使得该区域在光学影像中不再包含任何有效的地表观测信息。这意味着,对于这些区域而言,观测影像与真实地表之间不再存在可逆映射关系,地表信息无法通过单幅受云影响的影像直接恢复。

这种退化对遥感图像的信息表达造成了两方面的后果。首先,在空间维度上,云遮挡破坏了地物原有的连续结构,使道路、河流和地物边界等几何特征在影像中出现断裂或消失。其次,在光谱维度上,不同地物之间原本稳定的多光谱反射关系被云体反射特性所掩盖,导致光谱信息整体失真。由于云覆盖通常具有空间非均匀性,不同区域的退化程度存在显著差异,这进一步增加了云去除过程中结构恢复和光谱重建的难度。

\subsection{大气传输过程对光学遥感图像成像影响}

除云遮挡外,大气传输过程也是造成遥感图像退化的重要因素。即使在无明显云覆盖的情况下,地表辐射信号在传播过程中仍需穿过由气体分子、气溶胶和水汽组成的大气介质。这些成分会对电磁辐射产生吸收和散射作用,从而导致辐射能量衰减以及信号特性的改变。

从物理机制上看,大气吸收会降低辐射信号的整体强度,而散射过程则会引入额外的背景辐射成分。这种背景辐射叠加在地表反射信号之上,使得传感器接收到的信号同时包含目标信息和大气干扰信息。由于不同波段对大气成分的敏感程度不同,各光谱通道受到的影响并不一致,从而破坏了多光谱影像中原有的光谱一致性。

与云遮挡造成的信息“直接缺失”不同,大气传输引发的退化更多表现为信息质量的下降和不确定性的增加。一方面,大气状态具有明显的时空变化特性,其影响难以通过统一参数进行精确描述;另一方面,大气效应与地表反射特性之间往往呈现复杂的非线性关系,使得观测影像与真实地表之间的映射关系更加难以建模。这种不确定性会削弱影像的对比度和细节表达能力,使得细粒度结构信息和微弱光谱差异难以被准确辨识。

综合来看,云遮挡和大气传输共同作用,导致光学遥感影像在空间结构和光谱特征两个层面同时发生退化。其中,云遮挡主要引发信息的不可逆缺失,而大气传输则进一步加剧信息衰减和建模不确定性。这种双重退化机制使遥感图像云去除任务面临显著挑战,也为后续重建方法在结构恢复可靠性和光谱保真性方面提出了更高要求。

\section{SAR 图像成像原理与特性}

\subsection{SAR 图像成像原理}

SAR 图像是一种基于微波主动发射与回波接收的成像系统。传感器向地表发射微波信号,并接收由地物散射返回的回波信息,通过合成孔径技术在飞行方向上获得较高的空间分辨率。与依赖太阳辐射的光学成像不同,SAR 图像成像过程不受光照条件限制,且微波波段对云层和大气的穿透能力较强,因此能够在多云、阴雨等复杂气象条件下持续获取地表观测数据。

由于成像机理的不同,SAR 图像所表达的信息与光学影像存在显著差异。SAR 回波强度主要由地表目标的几何形态、电磁散射特性以及空间取向共同决定,而不是由地物的光谱反射特性主导。因此,SAR 图像对地表目标的表达更偏向于结构层面,尤其对具有明显几何形态的目标较为敏感。

在实际成像结果中,这种结构敏感性表现为对空间轮廓和连续形态的良好刻画。例如,道路、水体边界以及建筑物轮廓等具有明确几何结构的地物,在 SAR 图像中通常能够保持较好的空间连续性。即使在光学影像中被厚云完全遮挡的区域,这类结构信息在 SAR 图像中仍然往往是可观测的。这说明,相较于光学影像在云遮挡条件下面临的结构“信息缺失”,SAR 图像能够提供相对稳定的结构观测结果。

上述成像特性使 SAR 在云遮挡场景下具备显著优势:其能够在光学信息严重缺失的区域,为地表结构提供可靠的空间约束。这一优势为遥感图像云去除任务提供了新的信息来源,使得重建过程不再完全依赖光学影像中残存的局部纹理或统计先验。

然而,需要值得注意的是,SAR 图像并不直接包含光学影像所具备的光谱信息,其成像结果主要反映结构和物理属性,而非颜色和光谱特征。因此,SAR 更适合作为结构辅助信息,用于补充光学影像中缺失的空间结构,而难以单独完成光学影像的高质量重建。

(TODO:sar和光学图像展示,以及SAR的噪声影响)

\subsection{SAR 图像的相干斑噪声影响}

尽管 SAR 在结构信息获取方面具有明显优势,但其成像过程不可避免地会引入相干斑噪声。相干斑噪声产生于微波在同一分辨单元内与多个散射体发生相干叠加,其结果是在图像中形成具有随机性的颗粒状纹理。这种噪声并非简单的加性干扰,而是一种与回波强度相关的乘性噪声,其统计特性较为复杂。

从成像结果来看,相干斑噪声在一定程度上增强了结构边缘的对比度,使部分轮廓更加明显,但同时也在局部区域引入了强烈的随机波动。这种随机性破坏了图像的平滑性和一致性,使得真实结构信息与噪声成分在空间尺度上相互混叠。在高分辨率遥感场景中,这一问题尤为突出。

在云去除等图像重建任务中,如果在缺乏有效约束的情况下将 SAR 图像直接引入光学影像重建过程,噪声成分可能被误识别为可靠结构信息,从而在重建结果中引入不必要的纹理干扰,甚至影响光学影像原有的光谱一致性。此外,相干斑噪声的强度和分布受成像角度、地物类型及极化方式等因素影响,在不同场景中表现出较大的不稳定性,这进一步增加了 SAR 信息在跨模态利用过程中的不确定性。

因此,尽管 SAR 图像在云遮挡条件下能够提供稳定的结构观测能力,其噪声特性决定了该类信息不宜被无选择地使用。如何在充分利用 SAR 结构优势的同时,有效抑制相干斑噪声的负面影响,是多源遥感数据协同建模中必须重点考虑的问题。

\section{深度学习理论基础}

前述分析表明,在云遮挡条件下,光学遥感影像面临严重的信息缺失,而 SAR 图像虽能够提供稳定的结构观测能力,但其成像结果在信息表达形式和噪声特性上与光学影像存在显著差异。这种差异决定了云去除重建过程不仅需要补充缺失信息,还需要在多源信息之间实现有效选择与协同建模。因此,重建模型必须具备较强的非线性表达能力,能够在复杂退化条件下挖掘不同尺度、不同语义层级的有效特征。

深度学习模型凭借其端到端建模能力和层级化特征表达机制,为遥感图像重建任务提供了新的解决思路。然而,不同模型结构在特征建模范围、信息聚合方式以及计算复杂度等方面存在显著差异,其适用性和局限性有必要从理论层面进行分析。基于此,下面将从卷积神经网络、注意力机制以及 Transformer 等方面,对深度学习模型在图像重建任务中的建模特性进行系统阐述。

\subsection{卷积神经网络}

卷积神经网络(Convolutional Neural Network, CNN)是一类专门用于处理图像数据的深度学习模型,其设计目标是从原始像素空间中逐层提取具有判别能力的空间特征,并建立输入图像与目标输出之间的映射关系。由于图像具有显著的空间结构和局部相关性特征,CNN 在模型设计上显式引入了针对空间数据的建模假设,使其能够更高效地学习图像中的结构模式。与传统方法依赖人工设计特征不同,CNN 通过数据驱动的方式自动学习特征表示,在图像复原、修复与重建等低层视觉任务中展现出良好的适应性和泛化能力。

\subsubsection{卷积操作}

在图像处理中,一幅图像可以看作由像素点组成的规则二维网格,相邻或局部像素之间通常具有明显的统计相关性和结构连续性。卷积神经网络的卷积操作,正是基于这一先验假设进行设计的。

卷积层是卷积神经网络中最基本的组成单元,其核心计算单元为卷积核。卷积核可以理解为一个尺寸较小的权值矩阵,用于在图像的局部区域内对像素进行加权组合。对一个图像进行卷积时,通过滑动图像上的卷积核,在每一个位置对局部像素进行加权求和,从而将原始像素映射为对应的局部特征响应。这一过程可以理解为在局部邻域内对图像结构进行感知和编码,使模型能够有效捕获边缘、纹理以及局部形态变化等基础视觉特征。

此外,同一个卷积核会在整幅图像上重复使用,卷积层对相同的局部结构模式在不同位置具有一致的响应能力,这种机制称为权值共享。权值共享一方面显著减少了模型参数数量,降低了过拟合风险;另一方面使模型在不同空间位置上对相同结构模式具有一致的响应能力,从而增强了模型对平移变化的适应性。这一特性使 CNN 在高分辨率图像处理中具备较高的计算效率和稳定性。

通过设置多个不同的卷积核,卷积层可以同时提取多种类型的局部空间特征,为后续特征表示提供基础。

(TODO:卷积操作图)

\subsubsection{激活函数}

卷积运算从本质上来看是一种线性变换,仅依赖卷积层中的线性加权运算,模型的表达能力将受到较大限制。为增强网络对复杂映射关系的建模能力,CNN 通常在卷积运算之后引入激活函数,对特征进行非线性变换。激活函数的作用在于打破线性叠加的限制、增强模型的表示能力,使网络能够逼近更加复杂的非线性函数。

在图像重建任务中,退化过程往往涉及噪声干扰、信息缺失以及非线性失真等因素。非线性变换的引入,不再局限于简单的线性滤波或局部平滑操作,而是能够学习从退化图像到目标图像之间更为复杂的映射关系,为恢复细节结构和真实纹理提供必要的建模能力。

在实际应用中,常见的激活函数包括 Sigmoid、Tanh 以及 ReLU 等。其中,ReLU 函数因其形式简单、计算效率高、在一定程度上缓解梯度消失问题,而在图像重建任务中得到广泛应用。通过在多层网络中引入激活函数,CNN 能够逐步学习从退化图像到目标图像之间的复杂非线性关系,而不再局限于线性滤波或简单平滑操作。

(TODO:激活函数图)

\subsubsection{层级特征}

卷积神经网络通过多层卷积与非线性变换的逐级堆叠,形成层级化的特征表示机制。网络中不同层的特征并非人为设定,而是在训练过程中由数据驱动逐步学习得到的,其语义层级与特征所对应的空间尺度密切相关。

在网络的浅层阶段,特征由卷积核在高空间分辨率特征图上提取,更关注局部空间变化。这类特征主要响应于边缘、局部对比度变化以及细微纹理等局部结构信息,能够较好地刻画地物边界和细节特征。随着网络向更深层推进,卷积运算通常伴随下采样操作,使特征图的空间分辨率逐步降低,同时每个特征响应所对应的输入区域范围不断扩大,即有效感受野逐渐增大。

通过下采样和多层卷积的共同作用,浅层提取的局部结构信息在更大空间范围内被逐步组合和抽象,形成对更复杂空间模式的响应,即深层特征。相较于浅层特征,深层特征对整体空间结构和上下文关系更加敏感,而对局部细节变化的依赖相对减弱。

在卷积神经网络中,空间分辨率的变化是实现多尺度特征建模的重要手段。通过对特征图进行上下采样操作,网络能够在不同空间尺度上对图像信息进行建模,从而在局部细节刻画与全局结构感知之间取得平衡。

下采样操作的主要作用是降低特征图的空间分辨率,以扩大单个特征响应所对应的输入区域范围,即有效感受野。通过在较低分辨率的特征图上进行卷积运算,模型能够在更大空间范围内聚合上下文信息,从而提升对整体结构和长程空间关系的建模能力。在卷积神经网络中,下采样通常通过池化操作或步幅卷积实现。

与下采样相对应,上采样操作用于在重建阶段逐步恢复特征图的空间分辨率,使输出结果与输入图像尺寸一致。上采样可以通过插值、反卷积或其他特征重排方式实现,其核心目标是在恢复空间分辨率的同时,尽可能保留已学习的结构和语义信息。然而,由于下采样阶段已不可避免地丢失部分高频细节信息,单纯依赖上采样操作往往难以完全恢复精细结构。

因此,在图像重建任务中,上下采样操作通常需要与跨层特征传递机制相结合,通过融合不同尺度、不同分辨率的特征信息,在保证整体结构一致性的同时增强细节恢复能力。这种空间尺度变换与特征重组机制,为卷积神经网络在复杂场景下进行有效图像重建提供了重要支撑。

对于遥感图像云去除任务而言,这种层级特征表示具有重要意义。浅层特征有助于保留地物边界和局部细节,而通过下采样获得的深层特征则能够在更大空间尺度上建模地物结构和上下文信息,为在大面积云遮挡区域进行合理推断提供支持。通过在重建过程中合理利用不同层级的特征信息,CNN 能够在细节保真性与结构一致性之间取得平衡。

\subsubsection{局限性与发展需求}

尽管卷积神经网络在局部特征提取和细节恢复方面具有显著优势,但其建模过程主要依赖局部邻域信息的逐层传播,难以显式刻画图像中远距离区域之间的依赖关系。在云遮挡较为严重的场景中,光学遥感影像中往往缺乏可靠的局部参考信息,此时仅依赖 CNN 的局部建模机制,难以对大尺度结构一致性进行有效约束。

这一局限性表明,传统卷积神经网络在复杂信息缺失重建任务中仍存在不足,有必要进一步引入能够建模更大范围上下文关系的机制,以提升模型在遥感图像云去除任务中的重建可靠性。

\subsection{注意力机制}

注意力机制(Attention Mechanism)最初源于人类视觉与认知过程中的选择性关注行为,其核心思想是在大量信息中动态分配有限的建模能力,使模型能够优先关注对当前任务更为重要的部分。在深度学习框架下,注意力机制通过显式学习特征的重要性权重,对特征表示进行加权调节,从而实现对关键信息的突出建模与对冗余信息的抑制。这一机制为深度模型提供了一种超越固定结构计算的自适应特征选择能力。

在卷积神经网络中,特征提取主要依赖局部卷积运算和权值共享机制。尽管这种设计使模型在计算效率和局部结构建模方面具有显著优势,但也带来了固有的局限性。具体而言,标准卷积操作在同一层内对所有空间位置和特征通道采用统一的处理方式,默认各位置、各通道在特征表达中的重要性是等价的。这种“均匀建模”的假设在复杂视觉任务中往往并不成立,尤其是在存在遮挡、噪声或信息缺失的场景下,不同区域和不同特征对最终重建结果的贡献存在显著差异。

正是由于卷积操作难以显式建模这种特征重要性的差异,注意力机制被引入到卷积神经网络中,用以弥补其在特征选择层面的不足。通过在特征空间中学习一组自适应权重,注意力机制能够对原始特征进行重新分配,使模型在保持卷积结构优势的同时,具备对关键信息进行重点建模的能力。这一思想并不改变卷积的基本计算形式,而是在特征层面对卷积结果进行调制,因此具有良好的通用性和可插拔性。

根据建模维度的不同,现有注意力机制主要可以分为通道注意力和空间注意力两类。通道注意力侧重于刻画不同特征通道之间的重要性差异,其核心思想是通过建模通道间的相关性,增强对任务更有判别力的特征表示;空间注意力则关注特征图中不同空间位置的重要性分布,使模型能够更加聚焦于关键区域而忽略背景或干扰区域。在实际应用中,这两类注意力机制常被组合使用,通过从通道维度和空间维度对特征进行联合建模,实现更全面的特征增强。

在此基础上,通道—空间联合注意力模块逐渐成为图像重建任务中的主流选择。其中,典型代表如 SE(Squeeze-and-Excitation)模块主要通过通道维度建模特征重要性,而 CBAM(Convolutional Block Attention Module)则进一步引入空间注意力,对特征进行顺序或并行的多维度加权。这类模块在结构上相对轻量,能够在不显著增加计算复杂度的前提下,有效提升特征表示的判别性,因此在图像复原与重建任务中得到广泛应用。

(TODO:注意力模块图)

对于遥感图像云去除任务而言,注意力机制具有重要的理论意义。一方面,云遮挡导致图像中不同空间区域的信息完整性存在显著差异,注意力机制能够引导模型重点关注云覆盖区域及其结构边界,提高对关键缺失区域的建模能力;另一方面,在多源或多模态特征融合场景中,注意力机制可作为一种自适应调节手段,缓解不同特征之间信息分布不均或噪声干扰的问题,从而提升重建结果的稳定性与可靠性。然而,需要指出的是,注意力机制本质上仍然依附于局部特征建模框架,其作用主要体现在特征选择与增强层面,对建模范围的提升仍然有限。

\subsection{Transformer 理论基础}

随着深度学习在视觉领域的不断发展,研究者逐渐认识到,仅依赖卷积神经网络的局部建模机制,在复杂场景下难以充分刻画图像中大范围区域之间的关联关系。尤其是在存在大面积信息缺失或结构推断需求的任务中,模型不仅需要感知局部纹理和边缘信息,还需要在更大空间范围内建立全局一致性的结构约束。在此背景下,Transformer 模型应运而生,其核心目标是通过显式建模长程依赖关系,突破传统卷积结构在建模范围上的固有局限。

Transformer 最初提出于自然语言处理领域,用以解决循环神经网络在自然语言处理中长序列建模中存在的效率低、依赖建模受限等问题。其关键思想在于完全摒弃递归结构,转而采用基于注意力机制的全局特征交互方式,使序列中任意位置的特征都可以直接建立联系。这一思想随后被引入到视觉任务中,并逐渐发展为一类以全局建模为核心优势的通用特征建模框架。

与卷积神经网络相比,Transformer 在建模机理上存在本质差异。卷积神经网络通过局部卷积与逐层传播实现特征交互,其建模过程具有明显的局部性和层级性;而 Transformer 则通过自注意力机制在单层结构中直接建立全局依赖关系,使特征之间的交互不再受限于空间距离。这种差异使 Transformer 在捕获长程结构关系和全局上下文信息方面具有天然优势,但同时也带来了计算复杂度和数据需求方面的挑战。

Transformer 的核心组成单元是自注意力机制(Self-Attention)。自注意力的基本思想是:对于特征序列中的每一个元素,模型通过计算其与序列中所有其他元素之间的相关性,自适应地聚合全局信息,从而生成新的特征表示。在这一过程中,输入特征首先被映射为查询(Query, Q)、键(Key, K)和值(Value, V)三组表示。Q 用于刻画当前特征对其他特征的关注需求,K 用于描述各特征的属性,而 V 则承载被聚合的实际信息内容。通过计算 Q 与 K 之间的相似度,自注意力机制能够为不同位置分配不同的权重,并据此对 V 进行加权求和,实现全局范围内的信息交互。

(TODO 注意力机制图、多头、qkv)

为了进一步增强模型的表达能力,Transformer 通常采用多头注意力机制(Multi-Head Attention)。多头注意力是在自注意力机制基础上的一种扩展,使模型能够同时关注来自不同表示子空间(Representation Subspaces)和不同位置的信息。具体而言,该机制将输入的Q 、K 和 V 通过线性变换映射到 h 个独立的并行“头”(Head)中,也就是将特征映射到不同的表示子空间,在每个子空间中分别计算注意力权重,从而获得多种互补的特征交互结果,捕获特定的上下文依赖关系。最终,所有头的输出被拼接并通过线性投影融合,从而生成包含丰富语义和结构信息的特征表示,这种并行建模机制使模型能够从不同角度理解特征之间的关系。例如,不同的注意力头可以分别侧重于局部结构关联、长距离依赖关系或不同语义层级的特征交互。通过对多头注意力输出进行融合,模型能够在保持全局建模能力的同时,提升对复杂空间结构和多样依赖模式的表达能力。对于图像重建等需要同时兼顾局部细节与整体结构一致性的任务而言,多头注意力为模型提供了一种更加灵活且有效的全局特征建模方式。

除自注意力模块外,Transformer 还包含若干关键组件以保证模型的稳定性和表达能力。其中,前馈网络用于对注意力输出进行非线性变换,增强特征表达能力;残差连接和归一化操作则用于缓解深层结构中的训练困难,保证梯度传播的稳定性。这些组件共同构成了 Transformer 的基本计算单元,使其能够在保持全局建模能力的同时实现高效训练。

在视觉任务中,Transformer 通常需要结合位置编码或结构约束,以弥补其对空间结构感知能力不足的问题。通过引入显式或隐式的位置信息,Transformer 能够在全局建模的基础上保留一定的空间结构感知能力,从而更好地适应图像数据的特性。这一改进使 Transformer 在图像分类、分割以及图像重建等任务中展现出良好的潜力。

对于遥感图像云去除任务而言,Transformer 的理论优势主要体现在其对大范围上下文关系的建模能力。通过自注意力机制,模型能够在全局范围内整合结构信息,为大面积云遮挡区域的结构推断提供更强的约束。然而,由于 Transformer 在计算复杂度和数据依赖方面的特点,其在实际应用中往往需要与卷积结构或多尺度机制相结合,以在全局建模能力与计算效率之间取得平衡。上述特性为后续基于卷积与注意力协同设计的模型提供了重要的理论基础。

\subsection{CNN 与 Transformer 在图像领域的应用}

在计算机视觉领域的发展过程中,卷积神经网络长期作为主流模型架构被广泛应用于各类图像理解与重建任务。其成功主要得益于卷积操作对局部空间结构的高效建模能力以及较为成熟的网络设计范式。通过多层卷积、下采样与特征融合,CNN 能够逐级提取从局部纹理到高层结构的多尺度特征表示,在图像分类、目标检测、语义分割以及图像复原等任务中取得了稳定且可靠的性能表现。尤其在图像重建类问题中,卷积神经网络凭借对局部连续性和纹理一致性的良好建模能力,成为早期研究中最常采用的基础模型。

然而,随着任务复杂度的提升,研究者逐渐认识到仅依赖局部卷积运算难以充分刻画图像中远距离区域之间的关联关系。在需要全局上下文建模或大尺度结构推断的视觉任务中,CNN 的逐层局部传播机制在建模效率和表达范围上均受到一定限制。为缓解这一问题,部分研究通过引入注意力机制、多尺度结构或更深层的网络设计,对卷积模型进行改进,但其全局建模能力仍然受限于卷积操作的局部性假设。

在此背景下,Transformer 架构开始被引入到计算机视觉领域。Transformer 最初在自然语言处理任务中取得显著成功,其核心思想是通过自注意力机制显式建模序列中任意元素之间的依赖关系,从而突破传统模型在建模范围上的限制。Vision Transformer(ViT)是将 Transformer 架构直接应用于图像分类任务的代表性工作之一。该方法将输入图像划分为若干固定大小的图像块,并将其视作序列化的 Token 输入 Transformer 编码器进行处理,从而实现对整幅图像的全局建模。ViT 的提出表明,基于注意力机制的模型在视觉任务中同样具备较强的建模潜力。

在 ViT 之后,研究者针对 Transformer 在视觉任务中存在的计算复杂度高、对数据规模依赖较强以及空间结构建模能力不足等问题,提出了一系列改进模型。例如,引入局部窗口机制以降低计算开销、结合层级结构增强空间建模能力,或通过蒸馏与自监督策略提升训练效率。这些方法在保持 Transformer 全局建模思想的同时,对其结构形式进行了不同程度的调整,使其更好地适应图像数据的特点。

从整体发展趋势来看,卷积神经网络与 Transformer 在图像领域呈现出各自优势互补的特征。CNN 在局部结构建模、参数效率以及训练稳定性方面具有明显优势,而 Transformer 在全局依赖建模和长程关系刻画方面表现突出。因此,近年来部分研究开始探索将卷积结构与 Transformer 机制相结合的混合建模思路,以在局部细节建模与全局结构感知之间取得平衡。这类方法通过在不同层级或不同模块中引入不同的建模机制,为复杂视觉任务提供了更加灵活的特征建模方式。

对于遥感图像云去除与重建任务而言,图像中往往同时存在局部纹理缺失与大范围结构不完整等问题,不同模型结构在应对这些挑战时各具特点。卷积神经网络和 Transformer 在图像领域的演化与应用,为理解不同建模机制在复杂场景下的适用性提供了有益参考,其具体效果仍需结合任务特性和模型设计进行进一步分析。

\subsection{深度学习模型轻量化的理论基础}

\subsubsection{模型能力与复杂度}

从理论角度看,深度学习模型的建模能力来源于其对输入数据进行多层非线性变换和特征组合的能力。通过增加网络深度、拓宽特征通道维度或引入更复杂的特征交互机制,模型能够在高维特征空间中表示更加复杂的映射关系,从而提升对复杂模式的拟合能力。然而,这种能力提升通常伴随着模型复杂度的显著增长,使得模型在实际应用中面临计算与资源层面的约束。

模型复杂度通常可以从两个层面进行理解。一方面,复杂度体现在模型的参数规模上,即网络中可学习参数的数量,它直接决定了模型在存储和内存占用方面的需求;另一方面,复杂度还体现在计算复杂度上,通常以模型在一次前向传播过程中所需的计算量来衡量,该指标直接影响模型的推理时延和能耗。在实际应用中,这两种复杂度往往相互关联,共同决定了模型在特定硬件条件下的可部署性。

随着模型规模的不断增长,模型能力的提升往往依赖于更深的网络结构和更高维度的特征表示。在卷积神经网络中,增加网络深度和通道数虽然能够提升特征表达能力,但同时会导致卷积运算次数成倍增加,尤其是在高分辨率输入条件下,特征图尺寸较大,计算代价的累积效应尤为明显。在基于注意力机制或 Transformer 的模型中,模型能力的提升通常依赖于更大范围的特征交互,这类全局或半全局建模方式在理论上具有更强的表达能力,但其计算复杂度往往随特征维度和空间规模快速增长,使得模型在资源受限场景下面临较大挑战。

进一步来看,模型复杂度的增长并不完全等价于有效建模能力的提升。在深层网络中,不同特征通道或不同层级之间可能存在较高的信息冗余,部分计算对最终任务的贡献有限,但仍然参与了完整的前向计算过程。这种冗余计算在一定程度上提高了模型的表达上限,却同时显著增加了计算负担,使模型在推理阶段效率下降。此外,复杂度的过度增长还可能带来训练稳定性下降、模型过拟合风险增加以及对训练数据规模和硬件资源依赖加重等问题。

在遥感图像重建与云去除任务中,模型复杂度增长所带来的问题尤为突出。一方面,高分辨率遥感影像本身具有较大的空间尺寸和多通道特性,使模型在输入阶段即面临较高的计算压力;另一方面,为了应对复杂的退化机理和大范围信息缺失,模型往往需要具备较强的特征建模能力,从而进一步推高网络规模。这种对模型能力的需求与计算资源约束之间的矛盾,使得直接采用高复杂度模型在实际应用中存在一定局限性。

因此,在保证模型具备必要建模能力的前提下,如何控制模型复杂度、提升计算效率,成为模型设计中必须面对的问题。轻量化研究正是在这一背景下提出,其核心目标并非单纯压缩模型规模,而是通过识别模型中对任务贡献有限但计算代价较高的结构成分,对网络结构进行有针对性的优化,从而在性能与效率之间取得更加合理的平衡。这一思想对于遥感图像处理等对计算资源和推理效率具有较高要求的应用场景具有重要意义。

\subsubsection{常见轻量化网络结构设计思想}

在明确模型复杂度主要来源于网络结构设计之后,轻量化网络的研究逐渐从简单的参数压缩,转向对模型结构与学习过程的系统性优化。从理论角度看,轻量化设计的核心目标并非削弱模型的建模能力,而是通过减少冗余计算或引入有效约束,使有限的计算资源更多地集中于对任务具有关键作用的特征建模过程。

一种具有代表性的轻量化思想是基于通道维度的压缩与重组。在卷积神经网络中,特征通道数直接决定了特征空间的维度规模。尽管增加通道数能够提升模型的表达能力,但在实际网络中,不同通道之间往往存在较高的信息冗余。基于这一观察,研究者提出通过通道压缩或重标定的方式,降低高维特征映射中的冗余计算,从而在减少参数量和计算量的同时,尽量保留关键特征信息。

另一类重要的轻量化设计思想围绕卷积计算结构本身展开。标准卷积在空间维度和通道维度上同时进行特征交互,这在高分辨率和高通道数条件下会带来显著的计算开销。通过对卷积操作进行分解、分组或重构计算路径,可以在不改变卷积基本建模假设的前提下,有效降低计算复杂度。这类方法通过约束特征交互方式,实现了对计算代价的结构级控制,是轻量化卷积网络设计中的经典思路。

在深层网络结构中,特征在不同层级之间往往具有较强的相关性,部分中间特征在多个阶段中被重复计算。基于这一现象,一些轻量化方法通过增强特征重用来减少冗余计算,例如通过跨层连接或逐级特征精炼机制,使模型在保持网络深度的同时,避免对相似信息的反复建模,从而提升整体计算效率。

除结构级调整外,知识蒸馏为轻量化模型设计提供了一种从学习过程角度提升模型有效能力的策略。与通过压缩网络结构降低复杂度不同,知识蒸馏主要作用于模型训练阶段,其核心思想是在训练轻量模型时,引入一个性能较强的教师模型,对学生模型的学习过程施加额外约束。在具体机制上,知识蒸馏通常通过引导学生模型模仿教师模型的输出行为或中间特征表示来实现。教师模型在训练完成后,能够在高维特征空间中形成更加平滑且具有判别性的表示,其输出结果不仅包含最终预测信息,还隐含了不同类别或不同特征之间的相对关系。通过将这些信息作为软目标提供给学生模型,蒸馏过程使学生模型不再仅依赖于真实标签进行学习,而是同时受到教师模型行为的引导。从特征学习角度看,轻量模型由于结构容量受限,在训练过程中往往难以稳定地探索高维特征空间,容易陷入局部最优或学习到不充分的特征表示。教师模型所提供的软约束能够显式引导学生模型关注对任务判别更为关键的特征模式,从而减少无效特征的学习,提高训练效率和特征利用率。

随着注意力机制和 Transformer 结构在视觉任务中的应用,轻量化研究也逐渐扩展到全局特征交互过程的复杂度控制。由于全局自注意力在理论上具有较高的计算复杂度,部分研究通过限制注意力计算的空间范围、降低特征维度或引入分层建模策略,对全局建模过程进行约束。这类方法在保留关键依赖关系建模能力的同时,显著降低了注意力机制带来的计算开销,使相关模型在实际应用中更具可行性。

总体而言,现有轻量化方法既包括针对网络结构的复杂度削减,也包括通过知识迁移和学习约束提升轻量模型有效能力的策略。不同轻量化思想在削减计算开销和保持模型性能方面各有侧重,其适用性需要结合具体任务特性和应用场景进行综合考量。

\subsubsection{精度与效率的权衡}

在深度学习模型设计中,精度与效率是两个密切相关但往往相互制约的目标。精度通常用于衡量模型在特定任务上的预测或重建能力,在图像重建与云去除任务中,精度不仅体现在像素层面的误差大小,还反映在结构完整性、纹理一致性以及光谱保真性等多个方面。高精度模型通常能够更准确地恢复被遮挡区域的地物结构和细节特征,从而提高重建结果的可靠性。

效率则主要反映模型在实际运行过程中的资源消耗情况,通常包括推理阶段的计算量、模型参数规模、内存占用以及推理时延等因素。对于高分辨率遥感影像处理任务而言,效率直接决定了模型在有限计算资源条件下的可用性,尤其是在大规模数据处理或端侧部署场景中,模型的计算效率往往成为影响其实际应用价值的重要因素。

在实际应用中,精度与效率往往同时被提出要求。一方面,云去除和图像重建任务对结果质量具有较高要求,过低的精度会导致结构失真或细节缺失,从而影响下游应用的可靠性;另一方面,遥感影像数据规模大、分辨率高,若模型计算复杂度过高,将难以满足实际应用中的时效性和资源约束。因此,单纯追求高精度或极致效率,均难以满足实际需求。

精度与效率之间的冲突主要源于模型能力与复杂度之间的内在关系。提升模型精度通常依赖于更强的特征表达能力,而这往往通过增加网络深度、拓宽特征通道或引入复杂的特征交互机制来实现,这些设计会直接导致计算量和参数规模的增加。相反,为提升模型效率而削减网络结构,往往会限制模型的表达能力,使其难以充分刻画复杂场景中的细节和结构,从而对重建精度产生不利影响。

因此,模型轻量化的核心并非在精度和效率之间进行简单取舍,而是在两者之间寻找合理平衡。通常的做法是在满足任务最低精度需求的前提下,尽可能降低模型复杂度。例如,可以优先保留对重建质量贡献较大的关键结构,削减冗余计算较多但对性能提升有限的部分;或通过引入辅助约束和训练策略,在不增加推理阶段复杂度的情况下,弥补因结构压缩带来的精度损失。这种以任务需求为导向的权衡策略,使模型在性能和效率之间达到可接受的折中状态。

在遥感图像云去除任务中,这种权衡尤为重要。云遮挡条件复杂、地物类型多样,对模型精度提出了较高要求;同时,遥感影像通常具有较大的空间尺度和数据规模,对模型效率也构成了现实约束。因此,轻量化模型的设计需要在保证基本重建质量和结构一致性的前提下,控制计算开销和模型规模,使模型能够在实际应用环境中稳定运行。上述权衡原则为后续轻量化模型的结构设计与实验分析提供了重要的理论指导。

\subsection{遥感图像重建质量评价指标}

在遥感图像云去除与重建任务中,模型性能的优劣不仅体现在视觉效果上,还需要通过定量指标进行客观评估。由于云遮挡会导致像素缺失、结构破坏以及光谱信息失真,单一评价指标往往难以全面反映重建结果的质量。因此,合理选择和理解评价指标的物理含义与侧重点,是对不同方法进行公平比较和性能分析的重要前提。

为对遥感图像云去除与重建任务中的模型性能进行定量评估,通常需要从像素精度、结构一致性、光谱保真性以及误差幅度等多个角度对重建结果进行综合分析。基于上述考虑,本文采用峰值信噪比(Peak Signal-to-Noise Ratio, PSNR)、结构相似性指数(Structural Similarity Index Measure, SSIM)、光谱角映射(Spectral Angle Mapper, SAM)以及平均绝对误差(Mean Absolute Error, MAE)作为评价指标,对去云结果的重建质量进行系统评估。

设 $x$ 表示模型生成的去云光学影像,$y$ 表示对应的真实无云影像,$n$ 为图像中的像素总数,并且实验中所有影像都归一化至 $[0,1]$ 区间时,各评价指标的定义如下。

首先,PSNR 是图像质量评估中最常用的指标之一,其数值越大表示重建结果在像素层面越接近真实影像。PSNR 基于均方根误差(Root Mean Square Error, RMSE)计算,PSNR 计算公式为:
\begin{equation}
PSNR(x, y) = 20 \cdot \log_{10} \left( \frac{1}{RMSE(x, y)} \right)
\end{equation}

其中,RMSE 定义为:
\begin{equation}
RMSE(x, y) = \sqrt{\frac{1}{n} \sum_{i=1}^{n} (x_i - y_i)^2}
\end{equation}

为评估去云结果在感知层面的结构相似性,本文进一步采用 SSIM 指标。该指标从亮度、对比度和结构三个方面衡量两幅图像之间的相似程度。设 $\mu_x$ 和 $\mu_y$ 分别表示图像 $x$ 和 $y$ 的均值,$\sigma_x^2$ 和 $\sigma_y^2$ 表示方差,$\sigma_{xy}$ 表示协方差,$\epsilon_1$ 与 $\epsilon_2$ 为防止分母为零而引入的常数,则 SSIM 的计算公式为:
\begin{equation}
SSIM(x, y) = \frac{(2\mu_x \mu_y + \epsilon_1)(2\sigma_{xy} + \epsilon_2)}
{(\mu_x^2 + \mu_y^2 + \epsilon_1)(\sigma_x^2 + \sigma_y^2 + \epsilon_2)}
\end{equation}
SSIM 的取值范围为 $[0,1]$,其数值越大表示结构相似性越高。

考虑到遥感影像通常具有多光谱特性,仅依赖像素误差和结构指标难以全面反映光谱保持能力,本文引入 SAM 作为光谱一致性评价指标。SAM 通过计算预测光谱向量与真实光谱向量之间的夹角来衡量光谱形态的一致性,其定义如下:
\begin{equation}
SAM(x, y) = \cos^{-1} \left(
\frac{\sum_{i=1}^{n} x_i y_i}
{\sqrt{\sum_{i=1}^{n} x_i^2 \cdot \sum_{i=1}^{n} y_i^2}}
\right)
\end{equation}
SAM 值越小,表示光谱失真越小,去云结果在光谱层面越接近真实影像。

此外,本文采用 MAE 对像素级误差幅度进行补充评估。相较于均方误差,MAE 对异常值具有更好的鲁棒性,其定义为:
\begin{equation}
MAE(x, y) = \frac{1}{n} \sum_{i=1}^{n} |x_i - y_i|
\end{equation}

综合上述四项指标,可以从像素精度、结构一致性以及光谱保真性等角度对去云结果的重建质量进行全面评价,为后续实验结果分析提供评价依据。

\section{本章小结}

本章围绕遥感图像云去除任务所涉及的相关理论基础展开了系统阐述。从遥感图像与 SAR 图像特性出发,分析各自的优缺点。对深度学习模型中的卷积神经网络、注意力机制以及 Transformer 等结构进行梳理。并在此基础上,进一步从模型能力与复杂度的角度探讨了深度学习模型轻量化的理论背景,为后续轻量化模型设计提供了理论依据。最后介绍了遥感图像云去除中常用的评价指标,为后续实验结果的定量分析与方法对比提供了统一的评价基础。基于本章所述的理论分析,下一章将结合具体任务需求,进一步介绍所提出的遥感图像云去除模型及其网络结构设计。









