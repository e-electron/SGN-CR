


\chapter{相关理论基础}
\thispagestyle{others}
\pagestyle{others}
\xiaosi

\section{本章引言}

遥感图像云去除任务涉及复杂的成像退化过程与信息缺失问题,其重建效果不仅受制于云与大气对光学成像的物理干扰机理,还与不同遥感数据源的信息表达特性及模型建模能力密切相关。为从理论层面支撑后续方法设计,有必要对云遮挡条件下的遥感图像退化特性、多源遥感数据的成像机理差异,以及深度学习模型在图像重建任务中的建模特性进行系统梳理。同时,明确遥感图像重建质量的评价依据,也是开展定量实验分析的重要前提。基于此,本章围绕遥感图像云去除所涉及的相关理论基础展开,为后文提出的方法模型的提出奠定理论基础。

\section{光学遥感图像退化原理}

\subsection{云遮挡对光学遥感成像影响}

在光学遥感成像过程中,传感器所获取的影像并非地表真实反射特性的直接记录,而是地表辐射信号在穿过云层与大气介质后形成的观测结果。当成像路径中存在云或复杂大气条件时,地表辐射信号在传播过程中会发生显著改变,导致观测影像与真实地表之间产生偏差。这种偏差在遥感影像中表现为信息退化,其结果不仅体现在图像质量下降,更重要的是造成地表信息在空间结构和光谱维度上的缺失。

云遮挡是导致光学遥感图像退化的主要因素之一,其本质原因在于云层对电磁辐射传播路径的直接阻断。如\ref{fig:2-1}简略示意图所示,光学遥感成像依赖地表目标对太阳辐射的反射信号,而当云层位于传感器与地表之间时,地表反射辐射在传播过程中会被云体大量反射和散射,难以有效到达传感。此时,传感器接收到的辐射信号主要来源于云体本身,而非真实地表目标。

从成像机制角度看,云遮挡引发的退化并非简单的亮度变化,而是导致地表信息在观测层面被直接“替换”。在厚云覆盖区域,来自地表的辐射信号几乎完全被云层屏蔽,使得该区域在光学影像中不再包含任何有效的地表观测信息。这意味着,对于这些区域而言,观测影像与真实地表之间不再存在可逆映射关系,地表信息无法通过单幅受云影响的影像直接恢复。

这种退化对遥感图像的信息表达造成了两方面的后果。首先,在空间维度上,云遮挡破坏了地物原有的连续结构,使道路、河流和地物边界等几何特征在影像中出现断裂或消失。其次,在光谱维度上,不同地物之间原本稳定的多光谱反射关系被云体反射特性所掩盖,导致光谱信息整体失真。由于云覆盖通常具有空间非均匀性,不同区域的退化程度存在显著差异,这进一步增加了云去除过程中结构恢复和光谱重建的难度。

\begin{figure}[h]
		\centering 
		\includegraphics[width=13cm]{chapters/figures/2-1.png}
	    \bicaption[\xiaosi 光学遥感图像成像简略图]{\wuhao 光学遥感图像成像简略图}{\wuhao Simplified diagram of optical remote sensing image formation}
	   	 \label{fig:2-1}
\end{figure}

\subsection{大气传输过程对光学遥感成像影响}

除云遮挡外,大气传输过程也是造成遥感图像退化的重要因素。即使在无明显云覆盖的情况下,地表辐射信号在传播过程中仍需穿过由气体分子、气溶胶和水汽组成的大气介质。这些成分会对电磁辐射产生吸收和散射作用,从而导致辐射能量衰减以及信号特性的改变。

从物理机制上看,大气吸收会降低辐射信号的整体强度,而散射过程则会引入额外的背景辐射成分。这种背景辐射叠加在地表反射信号之上,使得传感器接收到的信号同时包含目标信息和大气干扰信息。由于不同波段对大气成分的敏感程度不同,各光谱通道受到的影响并不一致,从而破坏了多光谱影像中原有的光谱一致性。

与云遮挡造成的信息“直接缺失”不同,大气传输引发的退化更多表现为信息质量的下降和不确定性的增加。一方面,大气状态具有明显的时空变化特性,其影响难以通过统一参数进行精确描述;另一方面,大气效应与地表反射特性之间往往呈现复杂的非线性关系,使得观测影像与真实地表之间的映射关系更加难以建模。这种不确定性会削弱影像的对比度和细节表达能力,使得细粒度结构信息和微弱光谱差异难以被准确辨识。

综合来看,云遮挡和大气传输共同作用,导致光学遥感影像在空间结构和光谱特征两个层面同时发生退化。其中,云遮挡主要引发信息的不可逆缺失,而大气传输则进一步加剧信息衰减和建模不确定性。这种双重退化机制使遥感图像云去除任务面临显著挑战,也为后续重建方法在结构恢复可靠性和光谱保真性方面提出了更高要求。

\section{SAR 图像原理与特性}

\subsection{SAR 图像成像原理}

SAR 图像是一种基于微波主动发射与回波接收的成像系统。传感器向地表发射微波信号,并接收由地物散射返回的回波信息,通过合成孔径技术在飞行方向上获得较高的空间分辨率。与依赖太阳辐射的光学成像不同,SAR 图像成像过程不受光照条件限制,且微波波段对云层和大气的穿透能力较强,因此能够在多云、阴雨等复杂气象条件下持续获取地表观测数据。

\ref{fig:2-4}展示了光学图像的无云图、有云图和对应的 SAR 图像,可以看到,由于成像机理的不同,SAR 图像所表达的信息与光学影像存在显著差异。SAR 回波强度主要由地表目标的几何形态、电磁散射特性以及空间取向共同决定,而不是由地物的光谱反射特性主导。因此,SAR 图像对地表目标的表达更偏向于结构层面,尤其对具有明显几何形态的目标较为敏感。

\begin{figure}[h]
    \centering 
    \includegraphics[width=13cm]{chapters/figures/2-4.png}
     \bicaption[\xiaosi 光学遥感图像无云图、有云图和对应 SAR 图像]{\wuhao 光学遥感图像无云图、有云图和对应 SAR 图像}{\wuhao Cloud-free images, cloud-covered images, and corresponding SAR images}
	\label{fig:2-4}
\end{figure}

在实际成像结果中,这种结构敏感性表现为对空间轮廓和连续形态的良好刻画。例如,道路、水体边界以及建筑物轮廓等具有明确几何结构的地物,在 SAR 图像中通常能够保持较好的空间连续性。即使在光学影像中被厚云完全遮挡的区域,这类结构信息在 SAR 图像中仍然往往是可观测的。这说明,相较于光学影像在云遮挡条件下面临的结构“信息缺失”,SAR 图像能够提供相对稳定的结构观测结果。

上述成像特性使 SAR 在云遮挡场景下具备显著优势:其能够在光学信息严重缺失的区域,为地表结构提供可靠的空间约束。这一优势为遥感图像云去除任务提供了新的信息来源,使得重建过程不再完全依赖光学影像中残存的局部纹理或统计先验。

然而,需要值得注意的是,SAR 图像并不直接包含光学影像所具备的光谱信息,其成像结果主要反映结构和物理属性,而非颜色和光谱特征。因此,SAR 更适合作为结构辅助信息,用于补充光学影像中缺失的空间结构,而难以单独完成光学影像的高质量重建。

\subsection{SAR 图像的相干斑噪声影响}

尽管 SAR 在结构信息获取方面具有明显优势,但其成像过程不可避免地会引入相干斑噪声。相干斑噪声产生于电磁波在分辨单元内与多个散射体发生相干叠加。当雷达波照射到地表时,在一个分辨单元内通常包含大量尺寸、形态与取向不同的散射体,各散射体返回的回波信号具有不同的幅度与相位。在复数域中,这些回波进行矢量叠加,其相位的随机性导致幅度出现随机波动,从而在强度图像中形成颗粒状纹理结构。

从数学建模角度看,SAR 强度图像通常可表示为乘性模型:$I(x) = R(x)\cdot N(x)$。其中,$I(x)$ 表示观测强度图像,$R(x)$ 为真实地表后向散射系数,$N(x)$ 为相干斑噪声分量。该表达式表明,相干斑噪声属于典型的乘性噪声,而非加性噪声,其幅度随地物散射强度变化而变化。

\begin{figure}[h]
    \centering 
    \includegraphics[width=10cm]{chapters/figures/2-5.png}
     \bicaption[\xiaosi SAR 受到相干斑噪声影响示意图]{\wuhao 光SAR 受到相干斑噪声影响示意图}{\wuhao Schematic diagram of the impact of speckle noise on SAR}
	\label{fig:2-5}
\end{figure}

根据成像处理方式不同,SAR 图像可分为单视和多视两类。单视 SAR 图像由一次复数回波直接形成,其强度图像通常服从指数分布,此时噪声方差较大,图像呈现强烈的颗粒状波动。为降低相干斑噪声影响,通常采用多视处理,即在距离向或方位向对多个独立观测结果进行平均。则多视强度图像近似服从 Gamma 分布,随着视数的增加,噪声方差显著降低,但与此同时,空间分辨率会相应下降。因此,多视处理本质上是一种分辨率与噪声之间的权衡机制。

从成像效果来看(\ref{fig:2-5}为模拟图),相干斑噪声在一定程度上增强了边缘区域的局部对比度,使部分结构轮廓更加突出,但同时也在均匀区域引入随机起伏,破坏图像的平滑性和一致性。在高分辨率遥感场景下,由于分辨单元更小,散射体数量减少,噪声波动更加显著,结构信息与噪声成分在空间尺度上高度混叠。

在云去除等跨模态重建任务中,如果缺乏有效约束机制而直接利用 SAR 特征,噪声成分可能被误识别为真实结构信息,从而在重建结果中引入不必要的纹理干扰,甚至影响光学影像的光谱一致性。此外,相干斑噪声的统计特性受成像角度、极化方式与地物类型影响,在不同场景下表现出显著差异,这进一步增加了跨模态协同建模的不确定性。

因此,尽管 SAR 图像能够在云遮挡条件下提供稳定的结构观测能力,其相干斑噪声特性决定了该类信息不宜被无选择地引入光学重建过程。在理论层面,多模态协同建模应同时满足两个基本要求:一是充分利用 SAR 的结构连续性优势;二是通过合理机制抑制相干斑噪声对光学结果的干扰。

\subsection{SAR 与光学信息的互补机理}

在遥感图像云去除任务中,引入 SAR 数据的理论依据来源于两类成像机制在信息表达层面的本质差异。光学遥感依赖太阳辐射反射获取地表光谱信息,其优势在于能够准确刻画地物的颜色特征和光谱差异,但在云遮挡条件下容易出现空间结构信息的不可逆缺失。相比之下,SAR 采用主动微波成像方式,其回波信号主要反映地表目标的几何形态、电磁散射特性及空间结构关系,对云层和复杂大气具有较强的穿透能力。因此,在厚云覆盖区域,尽管光学影像中的地表信息被完全遮蔽,SAR 图像仍然能够提供连续的空间结构观测结果。

从信息属性角度分析,光学影像与 SAR 影像在表达维度上具有显著互补性。光学影像以光谱反射为核心,强调不同地物之间的辐射差异,适于进行语义识别与光谱分析;而 SAR 影像则以散射强度为主导,强调几何轮廓与空间连续性,更有利于刻画道路、建筑物边界、水体轮廓等结构特征。这种“光谱主导”与“结构主导”的差异,使两种模态在信息空间上形成天然的互补关系。

进一步从退化机制角度分析,在云遮挡条件下,光学影像中的退化主要表现为地表辐射信息被云体反射信号替代,即空间结构与光谱信息同时缺失;而 SAR 成像过程不依赖可见光传播路径,其观测结果在相同区域通常保持结构连续性。因此,在跨模态建模框架下,SAR 可被视为一种结构先验信息源,用于约束光学影像中缺失区域的空间重建过程,从而提高结构恢复的稳定性与一致性。

然而,需要强调的是,SAR 并不直接包含光学影像中的光谱信息,其成像结果难以反映真实地物的颜色与光谱特征。因此,在多模态协同建模过程中,SAR 更适合作为结构约束信号,而光学分支仍需承担光谱重建与语义表达的主要任务。这种功能分工决定了跨模态融合策略应突出“结构引导而非光谱替代”的原则。

此外,由于 SAR 图像存在相干斑噪声,其结构信息与噪声成分在空间尺度上相互交织。若缺乏合理的约束机制,噪声成分可能被误当作有效结构特征引入重建过程,从而影响光学影像的平滑性与光谱一致性。因此,在理论层面上,多模态协同建模应同时满足两个基本要求:一是充分利用 SAR 提供的结构连续性优势;二是抑制其噪声对光学重建结果的干扰。

综上所述,SAR 与光学影像在成像机理与信息表达上的差异构成了遥感云去除任务中多模态协同建模的理论基础。SAR 提供结构约束,光学承担光谱表达,两者在信息维度上的互补关系为复杂退化条件下的稳定重建提供了可能。

\section{深度学习理论基础}

前述分析表明,在云遮挡条件下,重建模型必须具备较强的非线性表达能力,能够在复杂退化条件下挖掘不同尺度、不同语义层级的有效特征。而深度学习模型凭借其端到端建模能力和层级化特征表达机制,为遥感图像重建任务提供了新的解决思路。但不同模型结构在特征建模范围、信息聚合方式以及计算复杂度等方面存在差异,其适用性和局限性有必要从理论层面进行分析。基于此,下面将从卷积神经网络、注意力机制以及 Transformer 等方面,对深度学习模型在图像重建任务中的建模特性进行系统阐述。

\subsection{卷积神经网络}

卷积神经网络(Convolutional Neural Network, CNN)是一类专门用于处理图像数据的深度学习模型,其设计目标是从原始像素空间中逐层提取具有判别能力的空间特征,并建立输入图像与目标输出之间的映射关系。由于图像具有显著的空间结构和局部相关性特征,CNN 在模型设计上显式引入了针对空间数据的建模假设,使其能够更高效地学习图像中的结构模式。与传统方法依赖人工设计特征不同,CNN 通过数据驱动的方式自动学习特征表示,在图像复原、修复与重建等低层视觉任务中展现出良好的适应性和泛化能力\textsuperscript{\cite{he2016deep}}。

\subsubsection{卷积操作}

在图像处理中,一幅图像可以看作由像素点组成的规则二维网格(如\ref{fig:2-2}),相邻或局部像素之间通常具有明显的统计相关性和结构连续性。卷积神经网络的卷积操作,正是基于这一先验假设进行设计的。

卷积层是卷积神经网络中最基本的组成单元,其核心计算单元为卷积核。卷积核可以理解为一个尺寸较小的权值矩阵,用于在图像的局部区域内对像素进行加权组合。对一个图像进行卷积时,通过滑动图像上的卷积核,在每一个位置对局部像素进行加权求和,从而将原始像素映射为对应的局部特征响应。这一过程可以理解为在局部邻域内对图像结构进行感知和编码,使模型能够有效捕获边缘、纹理以及局部形态变化等基础视觉特征。

\begin{figure}[h]
		\centering 
		\includegraphics[width=12cm]{chapters/figures/2-2.png}
	    \bicaption[\xiaosi 卷积操作示意图]{\wuhao 卷积操作示意图}{\wuhao Convolution operation diagram}
	   	 \label{fig:2-2}
\end{figure}

此外,同一个卷积核会在整幅图像上重复使用,卷积层对相同的局部结构模式在不同位置具有一致的响应能力,这种机制称为权值共享。权值共享一方面显著减少了模型参数数量,降低了过拟合风险;另一方面使模型在不同空间位置上对相同结构模式具有一致的响应能力,从而增强了模型对平移变化的适应性。这一特性使 CNN 在高分辨率图像处理中具备较高的计算效率和稳定性。

通过设置多个不同的卷积核,卷积层可以同时提取多种类型的局部空间特征,为后续特征表示提供基础。

\subsubsection{激活函数}

卷积运算从本质上来看是一种线性变换,仅依赖卷积层中的线性加权运算,模型的表达能力将受到较大限制。为增强网络对复杂映射关系的建模能力,CNN 通常在卷积运算之后引入激活函数,对特征进行非线性变换。激活函数的作用在于打破线性叠加的限制、增强模型的表示能力,使网络能够逼近更加复杂的非线性函数。

在图像重建任务中,退化过程往往涉及噪声干扰、信息缺失以及非线性失真等因素。非线性变换的引入,不再局限于简单的线性滤波或局部平滑操作,而是能够学习从退化图像到目标图像之间更为复杂的映射关系,为恢复细节结构和真实纹理提供必要的建模能力。

在实际应用中,常见的激活函数包括 Sigmoid、Tanh 以及 ReLU 等。其中,ReLU 函数因其形式简单、计算效率高、在一定程度上缓解梯度消失问题,而在图像重建任务中得到广泛应用。通过在多层网络中引入激活函数,CNN 能够逐步学习从退化图像到目标图像之间的复杂非线性关系,而不再局限于线性滤波或简单平滑操作。

\subsubsection{层级特征}

卷积神经网络通过多层卷积与非线性变换的逐级堆叠,形成层级化的特征表示机制。网络中不同层的特征并非人为设定,而是在训练过程中由数据驱动逐步学习得到的,其语义层级与特征所对应的空间尺度密切相关。

在网络的浅层阶段,特征由卷积核在高空间分辨率特征图上提取,更关注局部空间变化。这类特征主要响应于边缘、局部对比度变化以及细微纹理等局部结构信息,能够较好地刻画地物边界和细节特征\textsuperscript{\cite{zeiler2014visualizing}}。随着网络向更深层推进,卷积运算通常伴随下采样操作,使特征图的空间分辨率逐步降低,同时每个特征响应所对应的输入区域范围不断扩大,即有效感受野逐渐增大\textsuperscript{\cite{luo2016understanding}}。

通过下采样和多层卷积的共同作用,浅层提取的局部结构信息在更大空间范围内被逐步组合和抽象,形成对更复杂空间模式的响应,即深层特征。相较于浅层特征,深层特征对整体空间结构和上下文关系更加敏感,而对局部细节变化的依赖相对减弱。

在卷积神经网络中,空间分辨率的变化是实现多尺度特征建模的重要手段。通过对特征图进行上下采样操作,网络能够在不同空间尺度上对图像信息进行建模,从而在局部细节刻画与全局结构感知之间取得平衡。

下采样操作的主要作用是降低特征图的空间分辨率,以扩大单个特征响应所对应的输入区域范围,即有效感受野。通过在较低分辨率的特征图上进行卷积运算,模型能够在更大空间范围内聚合上下文信息,从而提升对整体结构和长程空间关系的建模能力。在卷积神经网络中,下采样通常通过池化操作或步幅卷积实现。

与下采样相对应,上采样操作用于在重建阶段逐步恢复特征图的空间分辨率,使输出结果与输入图像尺寸一致。上采样可以通过插值、反卷积或其他特征重排方式实现,其核心目标是在恢复空间分辨率的同时,尽可能保留已学习的结构和语义信息。然而,由于下采样阶段已不可避免地丢失部分高频细节信息,单纯依赖上采样操作往往难以完全恢复精细结构。

因此,在图像重建任务中,上下采样操作通常需要与跨层特征传递机制相结合,通过融合不同尺度、不同分辨率的特征信息,在保证整体结构一致性的同时增强细节恢复能力\textsuperscript{\cite{lin2017feature}}。这种空间尺度变换与特征重组机制,为卷积神经网络在复杂场景下进行有效图像重建提供了重要支撑。

对于遥感图像云去除任务而言,这种层级特征表示具有重要意义。浅层特征有助于保留地物边界和局部细节,而通过下采样获得的深层特征则能够在更大空间尺度上建模地物结构和上下文信息,为在大面积云遮挡区域进行合理推断提供支持。通过在重建过程中合理利用不同层级的特征信息,CNN 能够在细节保真性与结构一致性之间取得平衡。

\subsubsection{局限性}

尽管卷积神经网络在局部特征提取和细节恢复方面具有显著优势,但其建模过程主要依赖局部邻域信息的逐层传播,难以显式刻画图像中远距离区域之间的依赖关系\textsuperscript{\cite{wang2018non}}。在云遮挡较为严重的场景中,光学遥感影像中往往缺乏可靠的局部参考信息,此时仅依赖 CNN 的局部建模机制,难以对大尺度结构一致性进行有效约束。

这一局限性表明,传统卷积神经网络在复杂信息缺失重建任务中仍存在不足,有必要进一步引入能够建模更大范围上下文关系的机制,以提升模型在遥感图像云去除任务中的重建可靠性。

\subsection{注意力机制}

注意力机制最初源于人类视觉与认知过程中的选择性关注行为,其核心思想是在大量信息中动态分配有限的建模能力,使模型能够优先关注对当前任务更为重要的部分。在深度学习框架下,注意力机制通过显式学习特征的重要性权重,对特征表示进行加权调节,从而实现对关键信息的突出建模与对冗余信息的抑制。这一机制为深度模型提供了一种超越固定结构计算的自适应特征选择能力。

在卷积神经网络中,特征提取主要依赖局部卷积运算和权值共享机制。尽管这种设计使模型在计算效率和局部结构建模方面具有显著优势,但也带来了固有的局限性。具体而言,标准卷积操作在同一层内对所有空间位置和特征通道采用统一的处理方式,默认各位置、各通道在特征表达中的重要性是等价的。这种“均匀建模”的假设在复杂视觉任务中往往并不成立,尤其是在存在遮挡、噪声或信息缺失的场景下,不同区域和不同特征对最终重建结果的贡献存在显著差异。

正是由于卷积操作难以显式建模这种特征重要性的差异,注意力机制被引入到卷积神经网络中,用以弥补其在特征选择层面的不足。通过在特征空间中学习一组自适应权重,注意力机制能够对原始特征进行重新分配,使模型在保持卷积结构优势的同时,具备对关键信息进行重点建模的能力。这一思想并不改变卷积的基本计算形式,而是在特征层面对卷积结果进行调制,因此具有良好的通用性和可插拔性。

根据建模维度的不同,现有注意力机制主要可以分为通道注意力和空间注意力两类。通道注意力侧重于刻画不同特征通道之间的重要性差异,其核心思想是通过建模通道间的相关性,增强对任务更有判别力的特征表示;空间注意力则关注特征图中不同空间位置的重要性分布,使模型能够更加聚焦于关键区域而忽略背景或干扰区域。在实际应用中,这两类注意力机制常被组合使用,通过从通道维度和空间维度对特征进行联合建模,实现更全面的特征增强。

在此基础上,通道—空间联合注意力模块逐渐成为图像重建任务中的主流选择。其中,典型代表如 SE(Squeeze-and-Excitation)模块\textsuperscript{\cite{hu2018squeeze}}主要通过通道维度建模特征重要性,而 CBAM(Convolutional Block Attention Module)\textsuperscript{\cite{woo2018cbam}}则进一步引入空间注意力,对特征进行顺序或并行的多维度加权。这类模块在结构上相对轻量,能够在不显著增加计算复杂度的前提下,有效提升特征表示的判别性,因此在图像复原与重建任务中得到广泛应用。

\begin{figure}[h]
		\centering 
		\includegraphics[width=14cm]{chapters/figures/2-3-CBAM.png}
	    \bicaption[\xiaosi 通道–空间联合注意力示意图]{\wuhao 通道–空间联合注意力示意图CBAM 示意图}{\wuhao Channel–Spatial Attention}
	   	 \label{fig:2-3-CBAM}
\end{figure}

对于遥感图像云去除任务而言,注意力机制具有重要的理论意义。一方面,云遮挡导致图像中不同空间区域的信息完整性存在显著差异,注意力机制能够引导模型重点关注云覆盖区域及其结构边界,提高对关键缺失区域的建模能力;另一方面,在多源或多模态特征融合场景中,注意力机制可作为一种自适应调节手段,缓解不同特征之间信息分布不均或噪声干扰的问题,从而提升重建结果的稳定性与可靠性。然而,需要指出的是,注意力机制本质上仍然依附于局部特征建模框架,其作用主要体现在特征选择与增强层面,对建模范围的提升仍然有限。

\subsection{Transformer 理论基础}

随着深度学习在视觉领域的不断发展,研究者逐渐认识到,仅依赖卷积神经网络的局部建模机制,在复杂场景下难以充分刻画图像中大范围区域之间的关联关系。尤其是在存在大面积信息缺失或结构推断需求的任务中,模型不仅需要感知局部纹理和边缘信息,还需要在更大空间范围内建立全局一致性的结构约束。在此背景下,Transformer 模型应运而生,其核心目标是通过显式建模长程依赖关系,突破传统卷积结构在建模范围上的固有局限\textsuperscript{\cite{vaswani2017attention}}。

Transformer 最初提出于自然语言处理领域,用以解决循环神经网络在自然语言处理中长序列建模中存在的效率低、依赖建模受限等问题。其关键思想在于完全摒弃递归结构,转而采用基于注意力机制的全局特征交互方式,使序列中任意位置的特征都可以直接建立联系。这一思想随后被引入到视觉任务中,并逐渐发展为一类以全局建模为核心优势的通用特征建模框架。

与卷积神经网络相比,Transformer 在建模机理上存在本质差异。卷积神经网络通过局部卷积与逐层传播实现特征交互,其建模过程具有明显的局部性和层级性;而 Transformer 则通过自注意力机制在单层结构中直接建立全局依赖关系,使特征之间的交互不再受限于空间距离。这种差异使 Transformer 在捕获长程结构关系和全局上下文信息方面具有天然优势,但同时也带来了计算复杂度和数据需求方面的挑战。

Transformer 的核心组成单元是自注意力机制。自注意力的基本思想是:对于特征序列中的每一个元素,模型通过计算其与序列中所有其他元素之间的相关性,自适应地聚合全局信息,从而生成新的特征表示。在这一过程中,输入特征首先被映射为查询(Query, Q)、键(Key, K)和值(Value, V)三组表示。Q 用于刻画当前特征对其他特征的关注需求,K 用于描述各特征的属性,而 V 则承载被聚合的实际信息内容。通过计算 Q 与 K 之间的相似度,自注意力机制能够为不同位置分配不同的权重,并据此对 V 进行加权求和,实现全局范围内的信息交互。

(TODO 注意力机制图、多头、qkv)

为了进一步增强模型的表达能力,Transformer 通常采用多头注意力机制。多头注意力是在自注意力机制基础上的一种扩展,使模型能够同时关注来自不同表示子空间和不同位置的信息。具体而言,该机制将输入的Q 、K 和 V 通过线性变换映射到 h 个独立的并行“头”中,也就是将特征映射到不同的表示子空间,在每个子空间中分别计算注意力权重,从而获得多种互补的特征交互结果,捕获特定的上下文依赖关系。最终,所有头的输出被拼接并通过线性投影融合,从而生成包含丰富语义和结构信息的特征表示,这种并行建模机制使模型能够从不同角度理解特征之间的关系。例如,不同的注意力头可以分别侧重于局部结构关联、长距离依赖关系或不同语义层级的特征交互。通过对多头注意力输出进行融合,模型能够在保持全局建模能力的同时,提升对复杂空间结构和多样依赖模式的表达能力。对于图像重建等需要同时兼顾局部细节与整体结构一致性的任务而言,多头注意力为模型提供了一种更加灵活且有效的全局特征建模方式。

除自注意力模块外,Transformer 还包含若干关键组件以保证模型的稳定性和表达能力。其中,前馈网络用于对注意力输出进行非线性变换,增强特征表达能力;残差连接和归一化操作则用于缓解深层结构中的训练困难,保证梯度传播的稳定性。这些组件共同构成了 Transformer 的基本计算单元,使其能够在保持全局建模能力的同时实现高效训练。

在视觉任务中,Transformer 通常需要结合位置编码或结构约束,以弥补其对空间结构感知能力不足的问题。通过引入显式或隐式的位置信息\textsuperscript{\cite{liu2021swin}},Transformer 能够在全局建模的基础上保留一定的空间结构感知能力,从而更好地适应图像数据的特性。这一改进使 Transformer 在图像分类、分割以及图像重建等任务中展现出良好的潜力。

对于遥感图像云去除任务而言,Transformer 的理论优势主要体现在其对大范围上下文关系的建模能力。通过自注意力机制,模型能够在全局范围内整合结构信息,为大面积云遮挡区域的结构推断提供更强的约束。然而,由于 Transformer 在计算复杂度和数据依赖方面的特点,其在实际应用中往往需要与卷积结构或多尺度机制相结合,以在全局建模能力与计算效率之间取得平衡。上述特性为后续基于卷积与注意力协同设计的模型提供了重要的理论基础。

\subsection{模型轻量化理论基础}

\subsubsection{模型能力与复杂度}

深度学习模型的建模能力来源于多层非线性变换与高维特征组合能力。通过增加网络深度、扩大通道维度或引入更复杂的特征交互机制,模型能够表示更加复杂的映射关系,但这种能力提升通常伴随着模型复杂度的显著增长。

模型复杂度主要体现在参数规模与计算复杂度两个方面。参数量决定模型的存储与内存开销,而计算复杂度直接影响推理时延与能耗。在卷积神经网络中,增加层数和通道数会显著提升卷积计算量,尤其在高分辨率输入条件下,计算开销随特征图空间尺寸呈平方级增长。在基于注意力或 Transformer 的结构中,全局特征交互虽然增强了表达能力,但其计算复杂度随特征维度与空间规模快速增加\textsuperscript{\cite{dosovitskiy2020image}},在资源受限场景下尤为突出。

值得注意的是,模型复杂度的增长并不必然带来等比例的性能提升。深层网络中往往存在通道冗余与重复建模现象,部分计算对最终任务贡献有限,却增加了推理负担与训练不稳定风险。

在遥感图像重建与云去除任务中,这种能力与复杂度之间的矛盾更加明显。高分辨率、多光谱输入本身具有较大的计算规模,而复杂退化机理又要求模型具备较强的表达能力。因此,在保证必要建模能力的前提下,通过结构优化与计算约束实现性能与效率之间的平衡,成为模型设计中的关键问题。轻量化研究正是在这一背景下提出,其核心目标在于减少冗余计算,提升有效特征利用率,从而实现更高效的模型表达。

\subsubsection{轻量化方法的分类与基本思想}

轻量化设计的核心目标并非削弱模型的表达能力,而是在保证必要建模能力的前提下,通过减少冗余计算或引入有效约束,使有限的计算资源更加集中于对任务关键特征的建模过程。从实现层级上看,轻量化方法通常可以结构级和推理级进行优化。

结构级轻量化主要通过调整网络结构或算子形式,直接降低参数规模与计算复杂度。典型策略包括通道压缩与重标定\textsuperscript{\cite{hu2018squeeze}}、卷积算子分解或分组计算\textsuperscript{\cite{xie2017aggregated,howard2019searching}},以及特征重用机制等。这类方法通过约束特征交互路径或减少冗余通道,实现对计算开销的结构性削减。

训练与推理级优化则侧重于在不显著增加参数规模的前提下提升模型的有效表达能力。例如知识蒸馏\textsuperscript{\cite{hinton2015distilling}}通过教师模型引导轻量模型学习更具判别性的特征表示;而渐进式推理策略则通过多阶段逼近提升恢复精度。这类方法并不直接压缩模型结构,而是通过优化学习或推理过程,在效率与性能之间取得更合理的平衡。

在遥感图像重建场景中,由于输入分辨率高、退化机理复杂,上述两类轻量化思想往往需要协同使用:既通过结构优化降低计算负担,又通过推理机制弥补模型容量受限带来的性能损失。

\subsubsection{基于深度可分离卷积的结构级轻量化}

深度可分离卷积(Depthwise Separable Convolution, DSConv)是一种典型的结构级轻量化算子,其核心思想是将标准卷积在空间维度与通道维度上的耦合计算进行分解\textsuperscript{\cite{chollet2017xception,howard2019searching}}。具体而言,标准卷积在单次运算中同时完成空间特征提取与通道混合,而深度可分离卷积将其拆分为逐通道卷积(DWConv)与逐点卷积(PWConv)两个阶段,从而实现空间建模与通道建模的解耦。

从复杂度角度分析,设输入特征为 $H\times W\times C_{in}$,输出通道为 $C_{out}$,卷积核大小为 $k\times k$,则标准卷积参数量为
\begin{equation}
{Params}_{Conv}=k^{2} C_{in} C_{out}.
\end{equation}
深度可分离卷积参数量为
\begin{equation}
{Params}_{DS}=k^{2} C_{in}+C_{in} C_{out},
\end{equation}
其参数比为
\begin{equation}
\frac{{Params}_{DS}}{{Params}_{Conv}}=\frac{1}{C_{out}}+\frac{1}{k^{2}}.
\end{equation}
当 $C_{out}$ 较大时,深度可分离卷积能够显著降低参数规模与计算开销,且在高分辨率特征图条件下优势更加明显。

需要指出的是,DWConv 不进行跨通道混合,因此深度可分离卷积在降低复杂度的同时可能削弱通道交互能力。对于以局部空间建模为主的模块,其表达能力通常仍能满足需求;而在需要复杂语义组合的场景下,则需结合注意力机制或跨层融合结构进行补偿,以避免性能退化。

\subsubsection{基于递归细化的渐进式推理策略}

在图像恢复与重建任务中,渐进式方法是一类重要的优化思想,其核心目标是将复杂重建过程划分为多个逐步逼近的阶段,通过阶段间信息传递不断修正当前估计结果,从而提升最终恢复质量。与单次前向预测相比,渐进式机制能够更好地处理大范围结构缺失与局部细节误差共存的复杂退化场景。

根据实现形式的不同,渐进式方法可分为多阶段独立网络与参数共享递归细化两类。前者通常采用级联结构,每个阶段具有独立参数,通过逐级优化实现性能提升;后者则采用同一网络在多个阶段重复调用,通过参数共享的方式完成逐步逼近。

递归细化属于渐进式方法的一种典型实现形式。在该机制中,设恢复网络为 $F(\cdot)$,观测输入为 $y$,则递归过程可表示为:
\begin{equation}
x_{t+1}=F(x_t,y),
\end{equation}
其中 $x_t$ 表示第 $t$ 次迭代的估计结果。为增强训练稳定性与残差建模能力,更常见的形式为残差式递归:
\begin{equation}
x_{t+1}=x_t+F(x_t,y).
\end{equation}

由于各阶段共享同一组网络参数,模型参数规模不随迭代次数增加,但整体计算复杂度与推理时延通常随迭代步数 $T$ 近似线性增长。因此,递归细化机制提供了一种可调节的“性能与计算”权衡方式:在计算资源充足时可通过增加迭代次数提升重建精度,而在资源受限场景下则可减少阶段数以降低推理开销。

需要指出的是,递归细化并非迭代次数越多越好。当 $T$ 过大时,模型可能出现误差累积、过平滑或伪纹理增强等现象,反而导致重建性能下降。因此,在实际应用中需结合任务退化强度与计算预算合理设定迭代阶段数,并通过残差建模或置信度控制等机制增强训练与推理稳定性。

综上所述,递归细化是一种参数共享的渐进式推理机制,在不增加模型规模的前提下提升恢复能力,为轻量化模型在复杂场景中的性能补偿提供了有效路径。
\section{图像重建评价指标与训练目标}

\subsection{遥感图像重建质量评价指标}

在遥感图像云去除与重建任务中,模型性能的优劣不仅体现在视觉效果上,还需要通过定量指标进行客观评估。由于云遮挡会导致像素缺失、结构破坏以及光谱信息失真,单一评价指标往往难以全面反映重建结果的质量。因此,合理选择和理解评价指标的物理含义与侧重点,是对不同方法进行公平比较和性能分析的重要前提。

为对遥感图像云去除与重建任务中的模型性能进行定量评估,通常需要。在遥感图像云去除与重建研究中,常采用峰值信噪比(Peak Signal-to-Noise Ratio, PSNR)、结构相似性指数(Structural Similarity Index Measure, SSIM)、光谱角映射(Spectral Angle Mapper, SAM)以及平均绝对误差(Mean Absolute Error, MAE)作为评价指标,对去云结果从像素精度、结构一致性、光谱保真性以及误差幅度等多个角度进行综合评估。

设 $x$ 表示模型生成的去云光学影像,$y$ 表示对应的真实无云影像,$n$ 为图像中的像素总数,并且实验中所有影像都归一化至 $[0,1]$ 区间时,各评价指标的定义如下。

首先,PSNR 是图像质量评估中最常用的指标之一,其数值越大表示重建结果在像素层面越接近真实影像。PSNR 基于均方根误差(Root Mean Square Error, RMSE)计算,PSNR 计算公式为:
\begin{equation}
PSNR(x, y) = 20 \cdot \log_{10} \left( \frac{1}{RMSE(x, y)} \right)
\end{equation}

其中,RMSE 定义为:
\begin{equation}
RMSE(x, y) = \sqrt{\frac{1}{n} \sum_{i=1}^{n} (x_i - y_i)^2}
\end{equation}

为评估去云结果在感知层面的结构相似性,本文进一步采用 SSIM 指标。该指标从亮度、对比度和结构三个方面衡量两幅图像之间的相似程度。设 $\mu_x$ 和 $\mu_y$ 分别表示图像 $x$ 和 $y$ 的均值,$\sigma_x^2$ 和 $\sigma_y^2$ 表示方差,$\sigma_{xy}$ 表示协方差,$\epsilon_1$ 与 $\epsilon_2$ 为防止分母为零而引入的常数,则 SSIM 的计算公式为:
\begin{equation}
SSIM(x, y) = \frac{(2\mu_x \mu_y + \epsilon_1)(2\sigma_{xy} + \epsilon_2)}
{(\mu_x^2 + \mu_y^2 + \epsilon_1)(\sigma_x^2 + \sigma_y^2 + \epsilon_2)}
\end{equation}
SSIM 的取值范围为 $[0,1]$,其数值越大表示结构相似性越高。

考虑到遥感影像通常具有多光谱特性,仅依赖像素误差和结构指标难以全面反映光谱保持能力,本文引入 SAM 作为光谱一致性评价指标。SAM 通过计算预测光谱向量与真实光谱向量之间的夹角来衡量光谱形态的一致性,其定义如下:
\begin{equation}
SAM(x, y) = \cos^{-1} \left(
\frac{\sum_{i=1}^{n} x_i y_i}
{\sqrt{\sum_{i=1}^{n} x_i^2 \cdot \sum_{i=1}^{n} y_i^2}}
\right)
\end{equation}
SAM 值越小,表示光谱失真越小,去云结果在光谱层面越接近真实影像。

此外,本文采用 MAE 对像素级误差幅度进行补充评估。相较于均方误差,MAE 对异常值具有更好的鲁棒性,其定义为:
\begin{equation}
MAE(x, y) = \frac{1}{n} \sum_{i=1}^{n} |x_i - y_i|
\end{equation}

综合上述四项指标,可以从像素精度、结构一致性以及光谱保真性等角度对去云结果的重建质量进行全面评价,为后续实验结果分析提供评价依据。

\subsection{训练目标与损失函数设置}

遥感图像云去除任务不仅要求模型在像素层面准确重建被云遮挡区域的地物信息,还需同时保持地物结构的连续性与多光谱影像的光谱一致性。针对这一多目标约束问题,单一损失函数往往难以全面刻画去云结果的质量。为此,本文在训练阶段采用多项联合损失函数,对网络输出从像素精度、结构纹理以及光谱保真性三个层面进行综合约束。

考虑到该类联合损失形式已在遥感图像云去除与重建任务中得到广泛验证,本文在损失函数设计上未引入额外的复杂约束,而是基于已有研究中成熟且稳定的联合损失框架进行继承与应用。在本文中之后将采用文献 HPN-CR \textsuperscript{\cite{gu2025hpn}} 中的联合损失函数作为训练目标,其定义如下:
\begin{equation}
\mathcal{L}_{total}(P, T) 
= \alpha \mathcal{L}_{SmoothL_1}(P, T)
+ (1 - \alpha) \mathcal{L}_{MS-SSIM}(P, T)
+ \beta \mathcal{L}_{SAM}(P, T)
\label{eq:loss_total}
\end{equation}
其中,$P$ 与 $T$ 分别表示网络预测的去云光学影像与对应的真实无云影像,二者维度均为 $C \times H \times W$;$\alpha$ 与 $\beta$ 为用于平衡各损失项贡献的超参数。

(1)像素级重建损失($SmoothL_1$)

像素级重建损失用于直接约束模型输出在数值层面逼近真实无云影像,是云去除任务中最基础的监督信号。本文采用 $SmoothL_1$ 损失作为像素级约束项。相较于 $L_2$ , $SmoothL_1$ 在大误差区域更具鲁棒性,能够缓解云边缘和高反射区域可能带来的异常梯度,从而促进训练过程稳定收敛并提高像素层面的重建精度。

(2)结构相似性约束损失(MS-SSIM)

仅依赖像素级损失容易导致模型在云遮挡区域出现结构模糊或纹理断裂。为增强对结构与纹理细节的约束能力,本文引入多尺度结构相似性损失(Multi-Scale Structural Similarity, MS-SSIM),其从不同尺度对亮度、对比度与结构信息进行联合评估,有助于抑制重建过程中的过度平滑并提升纹理细节一致性。对应的损失形式为:
\begin{equation}
\mathcal{L}_{MS-SSIM}(P, T)
= 1 - {MS-SSIM}(P, T)
\label{eq:loss_msssim}
\end{equation}

(3)光谱一致性约束损失(SAM)

遥感影像云去除不仅要求结构恢复合理,还需尽可能保持地物的真实光谱特性。为此,本文引入光谱角映射(Spectral Angle Mapper, SAM)作为光谱一致性约束,通过最小化预测光谱向量与真实光谱向量之间的夹角,降低多光谱通道间的相对失真风险。其定义为:
\begin{equation}
\mathcal{L}_{SAM}(P, T)
= \cos^{-1} \left(
\frac{\sum\limits_{c,h,w} p_{c,h,w}\, t_{c,h,w}}
{\sqrt{\sum\limits_{c,h,w} p_{c,h,w}^2}\;
 \sqrt{\sum\limits_{c,h,w} t_{c,h,w}^2}}
\right)
\label{eq:loss_sam}
\end{equation}
其中,$p_{c,h,w}$ 与 $t_{c,h,w}$ 分别表示预测影像与真实影像在通道 $c$、空间位置 $(h,w)$ 的像素值。该约束能够有效抑制去云过程引入的颜色偏移与光谱形态失真,从而提升结果在后续遥感解译任务中的可用性。

通过式~(\ref{eq:loss_total}) 的联合优化,网络在训练过程中能够同时受到来自像素精度、空间结构与光谱分布三个层面的约束。在之后的实际训练中,本文参考相关工作经验并结合任务特点设置损失权重:为强化结构信息在去云重建中的约束作用,将 $\alpha$ 设置为 0.2,使结构相关的 MS-SSIM 项在整体优化中占据更高权重;同时将 $\beta$ 设置为 0.005,以保证 SAM 损失在数值量级上与其他损失项保持平衡。

\subsection{模型复杂度与推理效率评价指标}

除重建质量外,在资源受限或端侧部署场景中,模型复杂度与推理效率同样是重要评价维度。常见复杂度指标包括参数规模(Params)与浮点运算量(FLOPs),效率指标包括推理延迟(Latency)、帧率(FPS)及显存占用(Memory)。

Params 表示模型中所有可训练参数的总数,用于衡量模型的存储开销。

FLOPs 表示模型在单次前向传播过程中所需的浮点运算次数,用于刻画理论计算复杂度。对于卷积层,其 FLOPs 可表示为:
\begin{equation}
{FLOPs}_{conv} = 2 \times C_{in} \times C_{out} \times K^2 \times H \times W
\end{equation}
其中 $C_{in}$ 和 $C_{out}$ 分别表示输入与输出通道数,$K$ 为卷积核尺寸,$H$ 和 $W$ 为输出特征图的空间尺寸。

Latency 表示模型完成一次前向传播所需的时间;FPS 表示单位时间内可处理的图像数量,其关系为:
\begin{equation}
FPS = \frac{1000}{Latency(ms)}
\end{equation}

Memory 表示模型在推理阶段所消耗的 GPU 显存峰值,用于反映模型对硬件资源的实际需求。

通过将重建质量指标与复杂度指标结合分析,可以在保证重建精度的同时评估模型的计算开销,从而为不同算力约束条件下的模型设计与部署提供量化依据。

上述评价指标与训练目标构成了后续方法设计与实验分析的统一评价标准。第三章与第四章的实验结果均基于本节所定义的指标体系进行比较。

\section{本章小结}

本章围绕遥感图像云去除任务所涉及的相关理论基础展开了系统阐述。从遥感图像与 SAR 图像特性出发,分析各自的优缺点。对深度学习模型中的卷积神经网络、注意力机制以及 Transformer 等结构进行梳理。并在此基础上,进一步从模型能力与复杂度的角度探讨了深度学习模型轻量化的理论背景,为后续轻量化模型设计提供了理论依据。最后介绍了遥感图像云去除中常用的评价指标,为后续实验结果的定量分析与方法对比提供了统一的评价基础。基于本章所述的理论分析,下一章将结合具体任务需求,进一步介绍所提出的遥感图像云去除模型及其网络结构设计。









