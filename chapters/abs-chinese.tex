%中文摘要,自行编辑内容



\chapter{摘\quad 要}
\xiaosi

光学遥感影像在对地观测与资源环境监测中具有重要应用价值,但受云遮挡及复杂大气条件影响,影像中往往存在大面积信息缺失,严重制约了其在精细化遥感应用中的可靠性与实用性。合成孔径雷达(Synthetic Aperture Radar,SAR)影像具备全天时、全天候成像能力,能够在云遮挡条件下稳定获取地表结构信息,为光学遥感影像云去除提供了重要的辅助信息来源。然而,SAR 与光学影像在成像机理、空间分辨特性及信息表达形式上的显著差异,使得多模态信息的有效协同与融合仍面临较大挑战。

针对上述问题,本文围绕 SAR和光学协同的多模态遥感影像云去除任务,系统分析了遥感影像退化机理及信息缺失特性,提出了一种基于SAR 引导的遥感图像云去除网络(SAR-Guided Dual-Branch Network for Cloud Removal, SGN-CR)。该网络采用双分支编码架构,分别对 SAR 影像与含云光学影像进行特征建模,并以 SAR 分支提取的结构信息作为先验,引导光学分支的特征学习过程。通过在特征提取阶段引入 SAR 引导的注意力调制机制,SGN-CR 能够在云遮挡区域有效强化结构感知能力,抑制云干扰对光学特征建模的负面影响。

在此基础上,本文进一步设计了分层协同的跨模态特征融合策略,实现浅层结构约束与深层语义互补的有机结合,从而提升厚云遮挡条件下光学影像重建的稳定性与结构一致性。基于公开遥感数据集开展的大量对比实验结果表明,所提出的 SGN-CR 方法在峰值信噪比、结构相似性及光谱一致性等多项定量评价指标上均优于现有主流方法,尤其在大面积云遮挡场景下表现出更强的结构恢复能力与鲁棒性。进一步的消融实验验证了各关键模块设计的有效性。针对实际应用中对计算效率的需求,本文还提出了一种轻量化改进模型 Lite-SGN-CR,在显著降低模型参数量与计算复杂度的同时,保持了较为稳定的云去除性能。
\\

\noindent\songti\textbf{关键词:}遥感影像,云去除,SAR–光学多模态,结构引导,深度学习,轻量化

\clearpage