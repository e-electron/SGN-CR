

\chapter{总结与展望}
\thispagestyle{others}
\pagestyle{others}
\xiaosi

\section{主要工作总结}

遥感图像云去除是光学遥感数据应用中的关键问题之一,云遮挡会导致地物信息缺失和结构破坏,严重影响后续解译与分析任务的可靠性。针对传统光学方法在厚云覆盖条件下重建能力受限的问题,本文围绕多源遥感数据协同建模与深度学习方法展开研究,重点探讨了基于 SAR 与光学影像融合的遥感图像云去除方法及其轻量化设计。

本文系统分析了遥感图像在云遮挡和大气传输条件下的退化机理,明确了云去除任务在信息缺失、结构不连续以及光谱失真等方面所面临的主要挑战。在此基础上,结合 SAR 与光学遥感数据在成像机理和信息表达上的差异,阐述了利用 SAR 数据作为结构先验辅助光学影像重建的理论可行性,为后续模型设计提供了物理层面的支撑。

基于前文的分析,本文提出了一种基于 SAR 引导的多模态遥感图像云去除模型 SGN-CR。SGN-CR 通过构建双分支特征提取结构,分别对 SAR 与光学影像进行建模,并在特征层面引入跨模态引导与协同机制,使 SAR 图像中的稳定结构信息能够有效约束光学影像的重建过程。通过多层次特征融合与重建,模型在厚云遮挡条件下表现出较强的结构恢复能力和重建稳定性。实验结果表明,所提出方法在多项定量评价指标和视觉效果上均取得了较为理想的表现。

在此基础上,针对所提出模型在计算复杂度和实际部署方面的潜在限制,本文进一步开展了轻量化模型设计研究。通过对模型能力与复杂度关系的分析,结合轻量化网络结构设计思想,对原有模型结构进行了有针对性的优化,提出了轻量化版本模型。在保证重建性能基本不受明显影响的前提下,有效降低了模型参数规模和计算开销,提高了模型在资源受限场景下的应用可行性。

综上所述,本文围绕遥感图像云去除任务,从理论分析、模型设计到轻量化优化等多个层面展开研究,验证了多源遥感数据协同建模在复杂云遮挡条件下的有效性,并为后续相关研究提供了可参考的技术路线和实验基础。

\section{研究展望}

尽管本文围绕多源遥感图像云去除方法及其轻量化设计进行了系统研究,但受限于任务本身的复杂性和现有研究条件,仍存在若干值得进一步深入探索的问题。

(1)当前云去除方法在面对极端厚云遮挡或完全信息缺失区域时,重建结果仍不可避免地依赖模型的统计推断能力。未来研究可进一步关注如何对这类区域的重建结果进行可信度建模与不确定性刻画,使模型不仅能够给出重建结果,还能够反映不同区域预测的可靠程度,从而为后续遥感解译任务提供更安全的使用依据。

(2)现有方法大多在理想配准条件下开展研究,而在实际应用中,多源遥感数据往往存在分辨率差异、几何偏差或观测条件不一致等问题。如何在存在配准误差或模态差异的情况下,实现更加稳健的跨模态信息利用,是影响云去除方法工程可行性的重要问题,有待进一步研究。

(3)当前云去除模型的训练与评估主要依赖有限的数据集,其云型分布、地物类型和观测条件仍难以覆盖真实应用中的复杂情况。未来可通过引入更具代表性的真实观测数据,或探索弱监督、自监督等学习范式,降低模型对高质量成对数据的依赖,以提升方法在不同场景下的泛化能力。

(4)从应用角度看,云去除结果的价值最终体现在对下游遥感任务的支撑作用上。未来研究可进一步探讨云去除结果对地物分类、变化检测等任务的影响机制,并在此基础上构建面向应用目标的联合优化框架,使云去除过程更好地服务于实际遥感应用需求。

