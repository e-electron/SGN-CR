% 致谢
%\specialsectioning
\chapter{致 \quad 谢}
\thispagestyle{others}
\pagestyle{others}
\xiaosi

行文至此,心中感慨良多。忽而想起儿时盛夏老家的泥巴路,两旁种有高大的柏树,枝干挺拔,偶有一阵风过,树叶沙沙摩擦,投出的树荫伴着蝉鸣摇摆。晃动的斑驳树荫间一个小孩,坐着小板凳,支着的膝盖放着敞开的铁皮文具盒,嘴里飞速念着文具盒里印着的九九乘法表,也不觉耳边的聒噪蝉鸣讨厌。恍惚一梦,竟已匆匆二十年。二十年漫漫求学路,从小山村的蜿蜒泥巴路,到县城公交开过的水泥路,再到火车才能到的省城铁路,居然已经走出了这么远。

感谢一路上遇见的每一位老师,你们让我成为今天的我,你们不同的人生阅历,让我看到成长的意义。
感谢罗小波老师,三年前同样的春夏交接时,在你的办公室我们第一次相见,我成为你课题组的一员。你知识渊博但对我们从不苛责,在课题上给了我很大的自由和发挥空间。

感谢我的家人,你们从不过问太多,但总为我兜底。尤其感谢我的母亲,这二十年读书路,我若努力有一份、辛苦有一份,那你必是努力十份、辛苦十份。若不是你的坚持,我还留在老家的泥巴路上面,走不出,走不脱。你总是很豁达,在我迷茫低落时说出一句,“怕什么,还有你老娘呢。”你让我有勇气一步步走到今天,并且还将一步步走到更远。

感谢我的朋友们,一路上遇见你们是我的意外之喜。我很庆幸我有一群不管多久不联系都不生疏的朋友,不管我们走出去多远,都不会忘记拥有共同回忆的地方。感谢在我学业不顺时,为我出谋划策的你;感谢在我不善交际时,率先伸出手的你;感谢在我絮叨吐槽时,比我更先义愤填膺的你。感谢每一个出现在我生活中的你。

对自己和未来,心中有万千期盼与不安,皆化作六年级同样盛夏毕业时,郭老师给出的最后一句评语,“须知学如逆水行舟,不进则退。”愿自己,不要丢失学习的初心,不要丢失持续学习的勇气。坐在泥巴路板凳上的小孩,还会走到更多、更广阔的路上去。

最后,谨向参与本论文评审与审阅的各位老师致以诚挚的感谢,感谢你们在百忙之中对本文提出宝贵的意见与指导。




