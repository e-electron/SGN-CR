


\chapter{绪论}
\thispagestyle{others}
\pagestyle{others}
\xiaosi

\section{研究背景}

遥感技术是通过传感器在非接触条件下获取地表信息的重要技术手段,其基本原理是利用地物对电磁波的反射、辐射或散射特性,对地表目标进行成像与分析。相较于传统地面观测方式,遥感技术具备覆盖范围广、信息获取效率高以及能够长期、连续观测等显著优势,已广泛应用于国土资源调查、生态环境监测、农业生产评估、灾害应急响应以及全球变化研究等多个领域。随着对地观测系统的持续发展,遥感影像在空间分辨率、光谱维度和时间分辨率等方面不断提升,高分辨率、多光谱乃至高光谱遥感影像逐渐成为支撑精细化遥感应用的重要数据基础。

在众多遥感成像方式中,光学遥感影像因其能够直接反映地表物质的光谱反射特性,在地物识别、分类与定量分析等任务中具有不可替代的优势。多光谱光学影像不仅包含丰富的空间结构信息,还能够在不同波段上刻画植被、水体、裸地和人工建筑等典型地物的光谱差异,是当前遥感应用中最为常用的数据类型之一。然而,这种高度依赖地表反射辐射信息的成像机制,使光学遥感成像过程对大气条件极为敏感。云层、薄雾以及云影等大气现象会显著干扰传感器对地表的有效观测,导致成像质量下降甚至影像失效。统计结果表明\textsuperscript{\cite{king2013spatial}},全球陆地区域中相当比例的光学遥感影像在不同程度上受到云及云影的覆盖,该现象在热带及季风气候区尤为突出。云遮挡不仅会造成地表信息的局部缺失,还会引入明显的光谱畸变和结构模糊,从而严重影响后续遥感解译与定量分析任务的可靠性。

从成像机理角度来看,云层对光学遥感影像可用性的影响主要体现在多个层面。首先,云具有较强的反射与散射能力,会直接遮挡传感器对地表的观测视线,使云覆盖区域内的真实地表辐射信息完全缺失,形成典型的“信息空洞”区域。其次,云层引起的复杂大气散射效应会对云边缘或薄云区域的像素光谱造成污染,表现为亮度异常、色彩偏移或对比度降低,从而破坏地表物质原有的光谱一致性。此外,云影的存在会在地表形成非均匀的辐射衰减区域,进一步加剧影像内部的光照不一致性。这些因素共同作用,使受云影响的光学遥感影像在空间结构表达和光谱特性表征方面均出现显著失真,难以直接满足定量遥感分析与精细化应用的需求。

从遥感应用层面来看,云遮挡问题的影响并不仅限于单幅影像质量的下降,还会对多时相分析、变化检测以及长时间序列建模等任务产生连锁效应。在高云覆盖区域,能够满足应用需求的无云光学影像获取周期往往显著延长,甚至在关键时间窗口内完全缺失,从而削弱遥感数据在农业监测、灾害评估等时效性要求较高场景中的应用价值。因此,在受云污染的观测条件下实现地表信息的有效恢复,已成为提升光学遥感数据利用率和保障下游应用可靠性的基础性问题。

遥感图像云去除任务正是在上述背景下提出,其核心目标是在不依赖额外理想观测条件的情况下,对受云遮挡的光学影像进行信息重建。从问题本质上看,该任务属于一种高度不适定的信息缺失图像重建问题,其难点在于云覆盖区域内的真实地表信息往往被完全遮蔽,模型需要在缺乏直接观测的条件下推断合理的空间结构与光谱分布。尤其在厚云或大范围连续云覆盖场景中,地表纹理、边缘结构以及光谱分布均遭到严重破坏,传统基于局部统计或经验假设的方法难以获得稳定可靠的重建结果。这也使得云去除问题成为遥感图像处理领域中兼具理论挑战性和实际应用价值的重要研究方向。

随着遥感数据获取能力和应用需求的不断提升,云去除技术已不再仅服务于视觉层面的影像修复,而是逐渐演变为变化检测、时序分析和精细地物识别等高层遥感应用的关键前置环节。这对云去除结果在空间结构完整性、光谱保真性以及整体一致性等方面提出了更为严格的要求,也推动相关研究从传统的经验模型方法逐步向数据驱动的智能重建方法发展。

\section{国内外研究现状}

\subsection{单时相光学遥感图像云去除方法}

单时相光学遥感图像云去除方法是仅依赖单幅受云污染影像本身所包含的信息来进行地表重建,不引入额外的时间序列或多源辅助数据。由于其数据获取成本低、处理流程相对简洁,该类方法在早期遥感图像云去除研究中占据重要地位,至今仍在轻薄云覆盖场景下具有一定应用价值。根据方法建模思想的不同,单时相云去除技术大致可分为基于物理模型与先验假设的传统方法,以及基于深度学习的数据驱动方法。

\textbf{(1)基于物理模型与先验假设的传统方法}

传统单时相云去除方法通常基于对成像机理或统计特性的先验假设,认为云层与地表在空间分布、频率特性或物理成像参数上存在可区分的差异。典型方法包括空间域插值与相似像元替换方法、频率域滤波方法以及大气散射物理模型方法。

空间域方法 \textsuperscript{\cite{zhu2012modified,yu2011kriging,xu2019thin}}假设云覆盖区域与其邻域的云自由区域在统计特性或纹理结构上具有相似性,并且这种相似性有助于估计缺失数据以产生视觉上一致的无云图像,通过插值、回归或相似像元替换的方式对缺失区域进行估计。该类方法实现简单,但仅限于去除斑点云,在大范围连续云覆盖或复杂地物场景下,重建结果往往偏离真实地表语义。

频率域方法则利用薄云在低频成分中占优的特性,通过设计低通或多尺度滤波器分离云层与地表信息 \textsuperscript{\cite{wan2016removing}}。然而,该类方法对滤波参数高度敏感,不合适的频率可能导致无云区域中原始低频信息的丢失,引入细节损失或产生过度平滑现象。

基于物理模型的方法多借鉴大气散射模型,显式建模云层对辐射传输过程的影响,通过估计透射率和大气光参数恢复无云影像。通过精确估计和调整大气参数,这些方法可以有效地去除薄云。例如,基于先验的方法基于先验假设来估计透射图通过全局大气散射模型来恢复无云图像\textsuperscript{\cite{he2017thin}}。尽管该类方法具有较强的物理可解释性,但其有效性高度依赖于模型假设的准确性,在非均匀云分布或厚云场景下往往难以满足建模前提。

总体而言,传统方法在处理均匀薄云场景时具有一定优势,但受限于先验假设的表达能力,其在复杂云结构和语义重建方面存在明显瓶颈。

\textbf{(2)基于深度学习的单时相方法}

随着深度学习技术的发展,随着深度学习的不断进步,基于它的单图像去云已经获得了广阔的前景\textsuperscript{\cite{tao2022thick}}。研究者开始利用卷积神经网络(CNN)直接学习受云影像与无云影像之间的非线性映射关系。CNN 通过多层特征提取能够有效建模局部纹理和光谱模式,在薄云去除任务中取得了显著进展。为增强特征表达能力,部分研究引入多尺度结构\textsuperscript{\cite{yang2020multi,shao2019cloud}}、残差连接\textsuperscript{\cite{zhang2017beyond}}和注意力机制\textsuperscript{\cite{wen2022effective}},以提升细节恢复效果。

在此基础上,生成对抗网络(GAN)被引入单时相云去除任务中,通过对抗学习缓解对成对标注数据的依赖\textsuperscript{\cite{ma2023cloud,zhao2021cloud,toizumi2019artifact,wen2021generative}}。CycleGAN 等\textsuperscript{\cite{singh2018cloud,mo2022dca,ye2024cycle}}无监督框架通过循环一致性约束实现跨域映射,在一定程度上提升了真实场景下的泛化能力。然而,由于缺乏显式结构约束,GAN 类方法在大面积云遮挡区域仍容易产生伪影或色彩失真。

近年来,Transformer 架构凭借其全局建模能力被用于单时相去云任务\textsuperscript{\cite{ma2024sct}}。自注意力机制有助于捕获长程依赖关系,缓解 CNN 感受野受限的问题。然而,该类方法通常伴随着较高的计算复杂度,在高分辨率遥感影像场景下对算力和存储资源提出了较高要求。

尽管深度学习方法显著提升了单时相云去除的表现,但其本质仍受制于单幅影像的信息上限。当厚云完全遮挡地表区域时,网络难以从输入中推断缺失的真实结构与语义信息。

\textbf{(3)扩散模型方法}

扩散模型作为近年来兴起的一类生成式模型,凭借其强大的分布建模能力,在图像恢复与生成任务中展现出优异性能。部分研究尝试将条件扩散过程引入单时相或弱辅助云去除场景,通过逐步去噪的方式生成无云影像。

相较于 GAN,扩散模型在生成稳定性和细节一致性方面具有一定优势,但其多步迭代采样机制导致推理效率较低,难以直接适配大尺度、高分辨率遥感影像处理需求。此外,扩散模型对训练数据规模和计算资源的依赖较高,在工程部署和实时应用场景中难以真正应用。因此,尽管扩散模型在视觉生成任务中展现出潜力,但在高分辨率、多光谱遥感云去除任务中,其应用仍面临效率与稳定性方面的挑战。

综上所述,单时相云去除方法在无需额外数据的前提下具备良好的灵活性和适用性,但在厚云覆盖和复杂地物场景中,其重建能力受到信息缺失的根本限制。这一局限性促使研究逐步向引入多源辅助信息的方向发展。

\subsection{多时相光学遥感图像云去除方法}

为克服单时相光学影像在厚云覆盖区域中信息严重缺失的局限,研究者逐渐将注意力转向基于多时相光学影像的云去除方法。多时相遥感图像云去除方法利用同一区域在不同时间获取的多幅影像作为辅助信息\textsuperscript{\cite{zou2023spectral}},通过挖掘时间维度上的互补特征来恢复被云遮挡的地表区域。相比单时相方法,多时相方法在厚云或大面积连续云覆盖场景下具备更强的信息恢复能力,是解决严重信息缺失问题的重要研究方向之一。

多时相云去除方法的核心思想是利用地表目标在短时间尺度内相对稳定的假设,通过对多时相影像进行配准与融合,在云覆盖区域引入来自其他时相的无云或低云观测信息。根据是否依赖云掩膜信息,多时相云去除方法通常可分为基于掩膜的非盲方法与无需显式掩膜的盲方法。

(1)基于掩膜的多时相云去除方法

非盲多时相方法假设云覆盖区域可以通过人工标注或自动云检测算法准确获得,并利用云掩膜明确区分受损区域与有效区域\textsuperscript{\cite{lin2022robust,li2013principal}}。早期研究多采用相似像元替换、字典学习或优化模型\textsuperscript{\cite{lin2012cloud,cheng2014cloud,li2014recovering}},在多时相影像中搜索与云遮挡区域最相似的云自由块进行填充。

为缓解光谱差异和时间变化带来的影响,部分研究引入低秩建模、稀疏表示或张量分解方法,对多时相影像中的冗余信息进行建模,从而实现更稳健的缺失重建\textsuperscript{\cite{zhang2021thick,imran2022deep}}。这些方法在假设地表变化较小的前提下,能够较好地保持光谱一致性和结构连续性。

然而,非盲方法对云掩膜精度高度敏感,而复杂云形态和云影干扰使得云检测在实际应用中难以完全可靠。此外,当多时相影像间存在显著地物变化时,基于历史信息的替换策略容易引入结构错位或语义失真。

(2)多时相盲云去除方法

为降低对云掩膜的依赖,研究者提出了多时相盲云去除方法,将云检测与云去除过程统一建模\textsuperscript{\cite{lin2022robust}}。部分模型驱动方法通过矩阵分解或低秩—稀疏分离,将云成分视为稀疏噪声进行估计\textsuperscript{\cite{xu2016cloud,li2019cloud}}。该类方法具有一定理论可解释性,但对参数选择和数据分布较为敏感\textsuperscript{\cite{ji2022unified}}。

随着深度学习的发展,数据驱动的多时相方法逐渐成为主流。典型方法采用时序卷积网络或编码器—解码器结构,对多时相影像进行联合建模,以捕获时间维度上的相关性\textsuperscript{\cite{zou2023pmaa}}。部分研究进一步引入时间注意力机制,以增强对关键无云时刻的选择能力。

尽管深度学习方法在多时相场景中展现出较强的重建能力,但其性能高度依赖于时序数据的完整性和时间分布。当可用时相数量有限或时间间隔过长时,模型容易受到地表变化和配准误差的影响\textsuperscript{\cite{ji2020simultaneous}},导致重建结果不稳定。

(3)多时相方法的局限性分析

总体而言,多时相云去除方法通过引入时间冗余信息有效缓解了单时相信息不足的问题,在处理厚云和连续云覆盖方面具有明显优势。然而,该类方法在实际应用中仍面临多方面挑战:首先,多时相影像的获取受限于传感器重访周期,难以保证在关键时间节点获得高质量辅助数据;其次,地表变化、季节差异及成像条件不一致会破坏时序一致性假设;此外,多源时序数据的配准与预处理过程增加了系统复杂度。

上述问题在一定程度上限制了多时相方法在复杂场景下的稳定性与通用性,也促使研究逐步探索引入具有物理互补特性的多模态辅助信息,以进一步提升云去除的可靠性。

\subsection{SAR辅助光学遥感图像云去除方法}

合成孔径雷达(Synthetic Aperture Radar, SAR)作为一种主动微波成像方式,能够在全天时、全天候条件下获取地表信息\textsuperscript{\cite{gao2019cloud}},其成像过程几乎不受云层和光照条件的影响。这一独特优势使 SAR 数据在光学遥感图像云去除任务中具备重要的辅助价值\textsuperscript{\cite{chen2022cloud}}。研究者尝试通过引入 SAR 影像作为结构或纹理先验,来弥补光学影像在厚云遮挡条件下信息严重缺失的问题,从而显著提升云去除结果的完整性与稳定性。

需要指出的是,除将 SAR 作为辅助信息参与多模态融合外,部分研究也尝试仅依赖 SAR 数据直接恢复对应的光学影像,将该问题视为一种跨模态重建或图像翻译任务。此类方法通常基于深度生成模型学习 SAR 与光学影像之间的映射关系,在特定场景下能够生成具有一定视觉合理性的光学结果\textsuperscript{\cite{darbaghshahi2021cloud}}。

然而,从成像机理角度来看,SAR 与光学影像在物理基础和信息表达形式上存在显著差异。SAR 成像主要反映地表的几何结构、粗糙度和介电特性,而光学影像则以地物的光谱反射特性为主,两者之间并不存在严格的一一对应关系。这种模态不对称性使得仅凭 SAR 数据难以准确推断地表的真实光谱分布,尤其在植被覆盖、复杂地物混合或材质相近区域中,重建结果往往存在显著不确定性。

从实践角度看,基于 SAR 的单模态重建方法对训练数据分布高度敏感,其生成结果在跨区域或跨传感器场景下泛化能力有限。此外,SAR 影像中普遍存在的斑点噪声也容易在生成过程中被放大,进一步影响重建光学影像的稳定性和物理一致性。因此,单纯依赖 SAR 数据进行光学影像恢复在理论完备性和实际应用可靠性方面均存在明显局限。

基于上述分析,当前研究普遍倾向于将 SAR 数据作为结构先验或辅助信息,引导光学影像的云去除与重建过程,而非完全替代光学观测本身。这种多模态协同策略能够在保留光学影像光谱表达优势的同时,引入 SAR 提供的稳定结构信息,从而在复杂云遮挡场景下实现更为可靠的地表重建。

早期 SAR辅助的光学云去除研究多采用基于生成模型的跨模态映射策略,将云去除问题视为 SAR 到光学影像的条件生成或翻译任务。典型方法包括条件生成对抗网络(cGAN)及其变体,通过对抗训练学习 SAR 与光学影像之间的非线性映射关系\textsuperscript{\cite{li2020sar}}. 该类方法能够在一定程度上恢复被厚云遮挡的地表结构,在视觉效果上取得了较为理想的结果。

然而,基于生成式对抗学习的 SAR 辅助去云的方法通常将光学和 SAR 两种模态视为整体输入进行端到端映射,忽略了 SAR 与光学影像在成像机理、噪声分布以及信息表达形式上的本质差异。一方面,SAR 影像中普遍存在的斑点噪声容易在特征融合或生成过程中被放大,进而引入伪影;另一方面,纯生成式映射缺乏显式的结构约束,在复杂地物区域容易产生语义不一致或纹理失真现象。

因此部分研究引入注意力机制和多尺度特征融合策略,以增强关键区域的信息交互能力。例如,通过通道注意力或空间注意力引导网络重点关注云覆盖区域,从而提升云去除的针对性\textsuperscript{\cite{hao2023selecting}}。此外,一些方法尝试在网络中分别建模 SAR 与光学特征,再通过特征级融合实现跨模态协同\textsuperscript{\cite{xu2022glf}}。

尽管上述方法在一定程度上提升了 SAR 辅助云去除的效果,但现有融合策略仍普遍存在两个方面的不足。首先,多数方法在特征提取阶段采用先分别并行的提取 SAR 和光学图像特征结构,再在深层进行跨模态的信息交互,也就是说通常发生在网络的中后期。而这就导致 SAR 提供的结构先验未能对光学特征的早期建模形成有效约束。其次,部分方法在融合过程中对 SAR 特征采取简单叠加或全局注意方式,缺乏对不同层级特征语义差异的针对性建模,导致浅层噪声干扰和深层语义补全之间难以平衡。

近年来,Transformer 结构被引入 SAR 辅助的光学云去除任务中,以增强跨模态全局依赖建模能力\textsuperscript{\cite{li2023transformer}}。 自注意力机制能够在一定程度上缓解局部感受野限制问题,但其计算复杂度随特征分辨率呈平方增长,使得在高分辨率遥感影像场景中的应用面临显著的算力和存储压力。此外,纯基于全局注意的跨模态交互方式在缺乏显式结构引导的情况下,仍难以避免噪声信息的无差别传播。

综合来看,SAR 辅助光学的多模态云去除方法通过引入具有物理互补性的辅助信息,在厚云去除和结构恢复方面展现出明显优势。然而,如何在有效抑制 SAR 噪声干扰的同时,充分发挥其结构先验价值,并在不同语义层级实现合理的跨模态协同,仍是当前研究面临的关键问题\textsuperscript{\cite{zhang2020feature}}。这一问题也为后续基于结构引导的跨模态云去除模型设计提供了重要研究动机。

\section{论文研究主要内容}

针对光学遥感影像在云遮挡条件下面临的信息缺失与结构破坏问题,本文围绕 SAR 和光学多模态数据协同建模这一研究方向,系统开展遥感图像云去除方法研究。通过对现有单时相、多时相及多模态方法的分析可以发现,尽管引入辅助信息能够在一定程度上缓解云遮挡带来的影响,但在结构约束方式、跨模态协同机制以及模型实用性等方面仍存在不足。基于上述认识,本文的研究工作主要集中在以下三个方面:

(1)针对单时相方法信息不足、多时相方法稳定性受限的问题,本文从遥感成像机理和信息互补性的角度出发,将 SAR 数据引入光学遥感图像云去除任务中,重点研究如何利用 SAR 在厚云条件下仍然稳定存在的结构信息,为光学影像重建提供有效约束。通过对 SAR 与光学影像差异特性的分析,本文探索一种以光学影像重建为目标、以 SAR 结构信息为辅助的多模态云去除建模思路,为后续方法设计奠定基础。

(2)在多模态建模框架下,本文进一步关注跨模态信息如何在特征层面实现有效协同的问题。针对现有方法中 SAR 结构信息介入滞后、噪声干扰易传播以及不同层级特征语义差异未被充分利用等现象,本文从特征层级和信息属性的角度,对跨模态协同机制在云去除过程中的作用进行研究。通过分析不同层级特征在纹理抑制与结构补全中的功能差异,探讨在保证光谱一致性的前提下,引入结构先验以提升重建稳定性的有效策略。

(3)考虑到多模态深度学习模型在参数规模和计算复杂度方面面临的实际约束,本文在上述研究基础上进一步开展模型效率优化相关工作。通过分析多模态云去除模型中不同功能模块的计算开销与性能贡献关系,探索在尽量保持云去除质量和结构恢复能力的前提下,降低模型复杂度的可行方法,为模型在资源受限场景下的实际应用提供支持。

综上,本文围绕 SAR与光学多模态遥感图像云去除问题,从问题建模、跨模态协同机制分析以及模型实用性优化等层面展开研究,力求在复杂云覆盖场景下实现结构可靠、性能稳定且具有应用潜力的云去除方法。

\section{论文组织结构}

本文围绕遥感图像云去除问题,结合具有物理互补特性的多模态遥感观测数据,系统开展方法设计与应用研究。全文共分为五个章节,各章节内容安排如下。

第一章为绪论部分,主要介绍遥感图像云去除研究的背景与意义,对国内外相关研究现状进行系统梳理,并在此基础上明确本文的研究内容与总体研究思路,为后续章节奠定理论基础和研究框架。

第二章介绍本文研究所涉及的相关理论基础与关键技术,包括遥感成像基本原理、云遮挡对光学影像的影响特性以及深度学习在遥感图像重建中的应用,为后续方法设计提供必要的理论支撑。

第三章围绕多模态遥感图像云去除方法展开研究,重点针对厚云遮挡场景下光学影像结构信息严重缺失的问题,构建基于结构先验引导的云去除模型框架。通过对多模态特征协同建模与重建机制的系统分析,详细介绍模型整体设计、关键模块构成以及相应的实验验证结果,全面评估所提出方法在云去除精度和结构恢复方面的有效性。

第四章在第三章研究工作的基础上,进一步面向实际应用场景,对所提出模型进行轻量化与效率优化研究。通过对模型计算复杂度与性能表现的分析,探索在保证云去除效果的前提下,降低模型参数规模和计算开销的可行策略,并通过实验验证轻量化模型在性能与效率之间的平衡效果。

第五章是总结与展望。回顾了研究的核心工作,并对关键研究内容进行了总结。同时,还深入探讨了目前研究的局限性,并展望了未来研究的潜在发展路径

通过上述章节安排,本文从问题分析、方法设计到工程优化等多个层面,对多模态遥感图像云去除问题进行了系统研究。



\clearpage