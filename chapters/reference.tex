
%参考文献



\begin{thebibliography}{200}
\wuhao %设置参考文献字体大小
\linespread{1}\selectfont
\setlength{\itemsep}{-1.4ex} %缩小条目间行距
\thispagestyle{others}
\pagestyle{others}

\makeatletter
\renewcommand\@biblabel[1]{[#1]\hfill} %序号左对齐
\makeatother
\setlength{\labelsep}{0cm}

% 第一章
\bibitem{dosovitskiy2020image}
Dosovitskiy A. An image is worth 16x16 words: Transformers for image recognition at scale[J]. arXiv preprint arXiv:2010.11929, 2020.

\bibitem{liu2021swin}
Liu Z, Lin Y, Cao Y, et al. Swin transformer: Hierarchical vision transformer using shifted windows[C]//Proceedings of the IEEE/CVF international conference on computer vision. 2021: 10012-10022.

\bibitem{wang2020axial}
Wang H, Zhu Y, Green B, et al. Axial-deeplab: Stand-alone axial-attention for panoptic segmentation[C]//European conference on computer vision. Cham: Springer International Publishing, 2020: 108-126.

\bibitem{king2013spatial}
King M D, Platnick S, Menzel W P, et al. Spatial and temporal distribution of clouds observed by MODIS onboard the Terra and Aqua satellites[J]. IEEE transactions on geoscience and remote sensing, 2013, 51(7): 3826-3852.

% 单时相
\bibitem{zhu2012modified}
Zhu X, Gao F, Liu D, et al. A modified neighborhood similar pixel interpolator approach for removing thick clouds in Landsat images[J]. IEEE Geoscience and Remote Sensing Letters, 2011, 9(3): 521-525.

\bibitem{yu2011kriging}
Yu C, Chen L, Su L, et al. Kriging interpolation method and its application in retrieval of MODIS aerosol optical depth[C]//2011 19th international conference on geoinformatics. IEEE, 2011: 1-6.

\bibitem{xu2019thin}
Xu M, Jia X, Pickering M, et al. Thin cloud removal from optical remote sensing images using the noise-adjusted principal components transform[J]. ISPRS Journal of Photogrammetry and Remote Sensing, 2019, 149: 215-225.

\bibitem{wan2016removing}
Wan M, Li X. Removing thin cloud on single remote sensing image based on SWF[C]//2016 IEEE International Conference of Online Analysis and Computing Science (ICOACS). IEEE, 2016: 397-400.

\bibitem{he2017thin}
He M, Wang B, Sheng W, et al. Thin cloud removal method in color remote sensing image[J]. Opt. Tech, 2017, 43: 503-508.

\bibitem{tao2022thick}
Tao C, Fu S, Qi J, et al. Thick cloud removal in optical remote sensing images using a texture complexity guided self-paced learning method[J]. IEEE Transactions on Geoscience and Remote Sensing, 2022, 60: 1-12.

\bibitem{yang2020multi}
Yang Q, Wang G, Zhao Y, et al. Multi-scale deep residual learning for cloud removal[C]//IGARSS 2020-2020 IEEE International Geoscience and Remote Sensing Symposium. IEEE, 2020: 4967-4970.

\bibitem{shao2019cloud}
Shao Z, Pan Y, Diao C, et al. Cloud detection in remote sensing images based on multiscale features-convolutional neural network[J]. IEEE Transactions on Geoscience and Remote Sensing, 2019, 57(6): 4062-4076.

\bibitem{zhang2017beyond}
Zhang K, Zuo W, Chen Y, et al. Beyond a gaussian denoiser: Residual learning of deep cnn for image denoising[J]. IEEE transactions on image processing, 2017, 26(7): 3142-3155.

\bibitem{wen2022effective}
Wen X, Pan Z, Hu Y, et al. An effective network integrating residual learning and channel attention mechanism for thin cloud removal[J]. IEEE Geoscience and Remote Sensing Letters, 2022, 19: 1-5.

\bibitem{ma2023cloud}
Ma D, Wu R, Xiao D, et al. Cloud removal from satellite images using a deep learning model with the cloud-matting method[J]. Remote Sensing, 2023, 15(4): 904.

\bibitem{zhao2021cloud}
Zhao Y, Shen S, Hu J, et al. Cloud removal using multimodal GAN with adversarial consistency loss[J]. IEEE Geoscience and Remote Sensing Letters, 2021, 19: 1-5.

\bibitem{toizumi2019artifact}
Toizumi T, Zini S, Sagi K, et al. Artifact-free thin cloud removal using gans[C]//2019 IEEE International Conference on Image Processing (ICIP). IEEE, 2019: 3596-3600.

\bibitem{wen2021generative}
Wen X, Pan Z, Hu Y, et al. Generative adversarial learning in YUV color space for thin cloud removal on satellite imagery[J]. Remote Sensing, 2021, 13(6): 1079.

\bibitem{singh2018cloud}
Singh P, Komodakis N. Cloud-gan: Cloud removal for sentinel-2 imagery using a cyclic consistent generative adversarial networks[C]//IGARSS 2018-2018 IEEE International Geoscience and Remote Sensing Symposium. IEEE, 2018: 1772-1775.

\bibitem{mo2022dca}
Mo Y, Li C, Zheng Y, et al. DCA-CycleGAN: Unsupervised single image dehazing using dark channel attention optimized CycleGAN[J]. Journal of Visual Communication and Image Representation, 2022, 82: 103431.

\bibitem{ye2024cycle}
Ye H, Xiang H, Xu F. Cycle-Gan network incorporated with atmospheric scattering model for dust removal of martian optical images[J]. IEEE Transactions on Geoscience and Remote Sensing, 2024.

\bibitem{ma2024sct}
Ma J, Chen Y, Pan J, et al. SCT-CR: A synergistic convolution-transformer modeling method using SAR-optical data fusion for cloud removal[J]. International Journal of Applied Earth Observation and Geoinformation, 2024, 130: 103909.

% 多时相
\bibitem{zou2023spectral}
Zou Z, Chen L, Jiang X. Spectral–temporal low-rank regularization with deep prior for thick cloud removal[J]. IEEE Transactions on Geoscience and Remote Sensing, 2023, 62: 1-16.

\bibitem{lin2022robust}
Lin J, Huang T Z, Zhao X L, et al. Robust thick cloud removal for multitemporal remote sensing images using coupled tensor factorization[J]. IEEE Transactions on Geoscience and Remote Sensing, 2022, 60: 1-16.

\bibitem{li2013principal}
Li H, Zhang L, Shen H. A principal component based haze masking method for visible images[J]. IEEE Geoscience and Remote Sensing Letters, 2013, 11(5): 975-979.

\bibitem{lin2012cloud}
Lin C H, Tsai P H, Lai K H, et al. Cloud removal from multitemporal satellite images using information cloning[J]. IEEE transactions on geoscience and remote sensing, 2012, 51(1): 232-241.

\bibitem{}
Cheng Q, Shen H, Zhang L, et al. Cloud removal for remotely sensed images by similar pixel replacement guided with a spatio-temporal MRF model[J]. ISPRS journal of photogrammetry and remote sensing, 2014, 92: 54-68.

\bibitem{li2014recovering}
Li X, Shen H, Zhang L, et al. Recovering quantitative remote sensing products contaminated by thick clouds and shadows using multitemporal dictionary learning[J]. IEEE Transactions on Geoscience and Remote Sensing, 2014, 52(11): 7086-7098.

\bibitem{zhang2021thick}
Zhang Q, Sun F, Yuan Q, et al. Thick cloud removal for sentinel-2 time-series images via combining deep prior and low-rank tensor completion[C]//2021 IEEE International Geoscience and Remote Sensing Symposium IGARSS. IEEE, 2021: 2675-2678.

\bibitem{imran2022deep}
Imran S, Tahir M, Khalid Z, et al. A deep unfolded prior-aided RPCA network for cloud removal[J]. IEEE Signal Processing Letters, 2022, 29: 2048-2052.

\bibitem{xu2016cloud}
Xu M, Jia X, Pickering M, et al. Cloud removal based on sparse representation via multitemporal dictionary learning[J]. IEEE Transactions on Geoscience and Remote Sensing, 2016, 54(5): 2998-3006.

\bibitem{li2019cloud}
Li X, Wang L, Cheng Q, et al. Cloud removal in remote sensing images using nonnegative matrix factorization and error correction[J]. ISPRS journal of photogrammetry and remote sensing, 2019, 148: 103-113.

\bibitem{ji2022unified}
Ji T Y, Chu D, Zhao X L, et al. A unified framework of cloud detection and removal based on low-rank and group sparse regularizations for multitemporal multispectral images[J]. IEEE Transactions on Geoscience and Remote Sensing, 2022, 60: 1-15.

\bibitem{ji2020simultaneous}
Ji S, Dai P, Lu M, et al. Simultaneous cloud detection and removal from bitemporal remote sensing images using cascade convolutional neural networks[J]. IEEE Transactions on Geoscience and Remote Sensing, 2020, 59(1): 732-748.

\bibitem{zou2023pmaa}
Zou X, Li K, Xing J, et al. Pmaa: A progressive multi-scale attention autoencoder model for high-performance cloud removal from multi-temporal satellite imagery[J]. arXiv preprint arXiv:2303.16565, 2023.

% SAR辅助
\bibitem{gao2019cloud}
Gao J, Zhang H, Yuan Q. Cloud removal with fusion of SAR and Optical Images by Deep Learning[C]//2019 10th International Workshop on the Analysis of Multitemporal Remote Sensing Images (MultiTemp). IEEE, 2019: 1-3.

\bibitem{chen2022cloud}
Chen S, Zhang W, Li Z, et al. Cloud removal with SAR-optical data fusion and graph-based feature aggregation network[J]. Remote Sensing, 2022, 14(14): 3374.

\bibitem{li2020sar}
Li Y, Fu R, Meng X, et al. A SAR-to-optical image translation method based on conditional generation adversarial network (cGAN)[J]. Ieee Access, 2020, 8: 60338-60343.

\bibitem{hao2023selecting}
Hao Y, Jiang W, Liu W, et al. Selecting information fusion generative adversarial network for remote-sensing image cloud removal[J]. IEEE Geoscience and Remote Sensing Letters, 2023, 20: 1-5.

\bibitem{li2023transformer}
Li C, Liu X, Li S. Transformer meets GAN: Cloud-free multispectral image reconstruction via multisensor data fusion in satellite images[J]. IEEE Transactions on Geoscience and Remote Sensing, 2023, 61: 1-13.

\bibitem{zhang2020feature}
Zhang J, Zhou J, Lu X. Feature-guided SAR-to-optical image translation[J]. Ieee Access, 2020, 8: 70925-70937.

% 第二章


% 第三章
\bibitem{ebel2022sen12ms}
Ebel P, Xu Y, Schmitt M, et al. SEN12MS-CR-TS: A remote-sensing data set for multimodal multitemporal cloud removal[J]. IEEE Transactions on Geoscience and Remote Sensing, 2022, 60: 1-14.

\bibitem{gu2025hpn}
Gu P, Liu W, Feng S, et al. HPN-CR: Heterogeneous Parallel Network for SAR-Optical Data Fusion Cloud Removal[J]. IEEE Transactions on Geoscience and Remote Sensing, 2025.

\bibitem{darbaghshahi2021cloud}
Darbaghshahi F N, Mohammadi M R, Soryani M. Cloud removal in remote sensing images using generative adversarial networks and SAR-to-optical image translation[J]. IEEE Transactions on Geoscience and Remote Sensing, 2021, 60: 1-9.

\bibitem{xu2022attention}
Xu M, Deng F, Jia S, et al. Attention mechanism-based generative adversarial networks for cloud removal in Landsat images[J]. Remote sensing of environment, 2022, 271: 112902.

\bibitem{wang2023cloud}
Wang Y, Zhang B, Zhang W, et al. Cloud removal with SAR-optical data fusion using a unified spatial–spectral residual network[J]. IEEE Transactions on Geoscience and Remote Sensing, 2023, 62: 1-20.

\bibitem{grohnfeldt2018conditional}
Grohnfeldt C, Schmitt M, Zhu X. A conditional generative adversarial network to fuse SAR and multispectral optical data for cloud removal from Sentinel-2 images[C]//IGARSS 2018-2018 IEEE International Geoscience and Remote Sensing Symposium. IEEE, 2018: 1726-1729.

\bibitem{xu2022glf}
Xu F, Shi Y, Ebel P, et al. GLF-CR: SAR-enhanced cloud removal with global–local fusion[J]. ISPRS Journal of Photogrammetry and Remote Sensing, 2022, 192: 268-278.

\bibitem{han2023former}
Han S, Wang J, Zhang S. Former-CR: A transformer-based thick cloud removal method with optical and SAR imagery[J]. Remote Sensing, 2023, 15(5): 1196.

% 第四章

\bibitem{zhou2025gcepanet}
Zhou Q, Wang X, Fang J, et al. GCEPANet: A Lightweight and Efficient Remote Sensing Image Cloud Removal Network Model for Optical-SAR Image Fusion[J]. Information Fusion, 2025: 104090.

\end{thebibliography}

%\bibliographystyle{unsrt}
%
%
%% 此参考文献名为ref.bib文件
%
%\bibliography{ref}
%\thispagestyle{others}





\clearpage